\chapter*{Introduction}


\section{The theory of Hawkes processes}
    \subsection{A quick dive into point processes}
        Quickly introduce notation and vocabulary of point processes.
    \subsection{The original Hawkes process: self-exciting interactions}
        Present HP as self-exciting model. Applications. Interest.
        Basics as cluster, branching? 
        Simulation.
        Talk about multivariate.
    \subsection{How to modelise inhibition in Hawkes processes?}
        Introduce different ways of including inhibition (non-linear function, multiplicative inhibition, mean fields).
            Pros, cons, choices.
        Motivations for including inhibition

\section{Inference}
    \subsection{A bit of context and history}
        Present classical inference methods for self-exciting, both parametric, non parametric.
    \subsection{The difficulties of inferring inhibition}
        Talk about missing parametric. Present non-parametric.
        Difficulties of parametric
    \subsection{Establishing connections: inferring non-null interactions}
        Existing methods of inferring support of matrix in multivariate (lasso, l1) %Pat, bac
    \subsection{Something is amiss: how to account for missing data?}
        Present different types of missing info (jattering, thinning, superposition)
        Existing and difficulties.


\section{Contributions}
    \subsection{Inference for univariate Hawkes processes with inhibition}
        s
    \subsection{Estimating inhibtion and interaction networks in multivariate Hawkes processes}
        d
    \subsection{Spectral analysis for inference of noisy Hawkes processes}
        szd




