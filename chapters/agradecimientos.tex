\chapter{Agradecimientos}

Como lo dicta la tradición tras escribir un tan largo y arduo trabajo, esta sección es una pequeña impresión del agradecimiento que tengo por mucha gente. 
Esta fue la gente que me apoyo durante estos tres años de tesis y sin quienes no habría podido llevar a cabo una tal labor. 

Como muchos saben y otras se estarán dando cuenta, soy una persona que no sabe comenzar las cosas, tampoco terminarlas y mucho menos resumirlas. Sin embargo intentaré con todo mi ser hacer de esta lectura una experiencia amena.

Un punto antes de empezar: mis agradecimientos no tienen orden, ni pies ni cabeza. La lectura debe ser como una montaña rusa: esperen su turno, vean tranquilamente pasar a los demás y, cuando sea su turno, sosténganse bien, no saquen las manos del carrito y disfruten el viaje.


Quiero empezar agradeciendo a mis supervisores, quienes tuvieron la difícil tarea de soportarme durante 3 años. Arnaud gracias por tu carácter determinante, decisivo y tajante, ya sea para arreglar problemas o resolver crucigramas. Maxime y Anna, esta aventura empezó con ustedes cuando decidimos de una manera vagamente espontánea trabajar juntos. Anna gracias por no haber respondido a aquella candidatura que te envié ya que de otra manera nunca me habría aventurado en el mundo de los procesos de Hawkes. Gracias por todos los momentos conviviales, las largas discusiones de matemáticas, ecología. Mis experiencias preferidas contigo fue compartir nuestros viajes para las conferencias, sobre todo a Toulouse. Buena suerte en tu Master. Maxime gracias por hacer de tu bureau nuestro cuartel general de reuniones, de siempre sacar las galletas ocultas cuando nos veíamos a las tempranas 9 de la madrugada y sobre todo por tu rigor científico. Les agradezco porque siento que hacer un doctorado es un reto enorme y estoy contento de lo que logramos hacer juntos.

Le agradezco a mi jurado por haber aceptado evaluar mi trabajo. Gracias François Roueff y Gordon J. Ross por haber tomado el tiempo de leer mi manuscrito. Gracias Celine Duval y Sophie Donnet por haber aceptado ser mis examinadoras. 

Felix, te agradezco por haber sido un guía de los procesos de Hawkes y sobre todo de la teoría espectral. Recuerdo haberte conocido en tu posdoctorado en el laboratorio y como toda buena historia ahora continuamos a trabajar juntos esta vez para el mio. Gracias por siempre dar de tu tiempo para explicarme cosas, para quejarnos de aquel libro (que no mencionaré pero tu sabes de cual hablo) o simplemente para charlar.

Le agradezco a todos los miembros del laboratorio que animan la vida científica de esta universidad con sus personalidades tan variadas, tan únicas pero sobre todo siempre bondadosas. Gracias Claire por haberme integrado en este laboratorio desde mi época de pasante y por hacernos sentir a mi y a los otros doctorantes como verdaderos colegas. Suerte en tu nueva aventura. Gracias Stéphane por siempre tener una sonrisa, por siempre echar la plática y compartir tus conocimientos. Espero que estos años como vecinos de oficina, de escuchar música, risas y cantos hayan sido no muy desagradables. Gracias Antoine por siempre recordarnos a los doctorantes que hay que disfrutar de la vida y saber cuando dejar de trabajar. Gracias Charlotte por tu gentileza, por organizar el grupo de Hawkes. Fue muy agradable formar parte del grupo de profesores de L3 Estadística contigo. Gracias a los demás miembros Etienne, Anna BH, Jean-Patrick, Maud, Raphael, Erwan, Sylvain, Lorenzo, ETC ETC ETC. Le agradezco de la misma manera a los miembros de la administración Valerie,  Nathalie, Marouane ETC ETC. En particular, gracias Louise que me acompañaste de pasante y en mis primeros años de tesis. Adoré cada momento que pude platicar contigo, mil gracias por constantemente preguntar como podías ayudar, hiciste que mi vida laboral fuera más amena. Gracias Hugues, el Atlas que carga el mundo informático de este laboratorio a base de mails con un humor singular e hilarante. No hay nada más divertido que leer que no puedes hacer algo “inmediatamente” pero que en 1 hora estará listo. Si hubiera sido tan eficaz, habría probablemente publicado 77 papeles durante mi tesis. Gracias a todos los miembros de limpieza porque gracias a ustedes podemos trabajar en condiciones óptimas. BUSCAR NOMBRE DE LA SEÑORA DE NUESTRO BUREAU.


En toda aventura se conoce gente que nos acompaña y con quienes compartimos logros, felicidad, desesperación y sobre todo cafés… muchos cafés. Camila, no podría estar más feliz de haber estado en la misma oficina contigo, nada mejor que tener un pedacito de Latinoamérica tan cerca. Aprecio cada sesión de canto y música, de enviarnos memes, de pegar el rostro del otro en las sillas y más que nada la felicidad de gritar cuando nos volvemos a ver, no importa que hayan pasado unos días o varias semanas sin vernos. Espero seguir con nuestras sesiones de shopping, karaoke, chisme y comida.
Iqraa, gracias por esas pláticas, a veces tan filosóficas, en el quinto piso, por esa época de ir por cafés. Eres una persona con mucho corazón. Una de mis mejores memorias es todas las pláticas de chisme, amores y desamores, risas y llantos que compartimos los cuatro junto con Ariane. 
Ariane te he ido conociendo poco a poco y aprecio mucho el habernos acercado tanto con los años. 

Quiero agradecer a esas dos personas que conocí gracias a la Maestría. Claire, Clairounette, Clairicot, Ma chère Claire, soy tan pero tan feliz de haberte conocido, de haberte propuesto de pasar tu pasantía en mi oficina porque fue el inicio de una amistad que hasta hoy es de las más preciadas para mí. Tienes más energía que el Sol, siempre te veo corriendo de derecha a izquierda (a veces literalmente), ya sea para compartir unos juegos de mesa en invierno o la playa en verano (mil mil mil gracias por hacerme descubrir Royan). No importa donde estés, siempre estarás cerca mio. 
Ludovic espero que andes leyendo la versión en español de los agradecimientos. Ludo, eres alguien que fui conociendo más y más con los años, gracias por compartir tu manera de sobrellevar la vida con una sonrisa, con calma, con una aura tan chill. Te agradezco por siempre haber sabido bromear conmigo y por siempre compartir tu pasión por la cultura, la historia, y los bebés feos del arte sacro. Gracias por bailar conmigo en tantas noches al borde del río (no te preocupes ya no me duele el codo).

Gracias Alexis. Gracias por todas las noches comiendo tu deliciosa comida. Gracias por enseñarme el mundo de las fiestas techno, por haberme acompañado a tantas fiestas pop (por más que decías que las odiabas). Te quiero agradecer por todo el cuidado que me diste, por todo ese amor y amistad, por todos esos tragos que compartimos, esas pláticas, los tantos chismes, los grafos, los consejos, los abrazos. Acercarme a ti fue de las cosas mas bonitas que me sucedieron en el doctorado. Sigue viviendo con la pasión e intensidad con la que la techno te hace bailar.
Antonin, mi ciela, que gusto me da tenerte en mi vida, poder reírnos tanto con esos edits de las estrellas pop, el poder cantar a todo pulmón Beyoncé y que siempre estés ahí para escucharme.
Sin importar lo que pase, ustedes siempre ocupan un lugar especial en mi corazón y espero poder tenerlos cerca por años.

Le agradezco a una doctorante que es tan dedicada como feminista e hiperactiva: Adeline. No pense encontrar alguien que piensa y habla tan rápido como yo, nuestras platicas en mi oficina (o la del jefe) eran caóticas, divertidas y llenas de consejos amorosos. Muchas gracias porque supiste venir y darme apoyo cuando lo necesité tanto. Aún recuerdo ver Batman juntos aquella noche.

Hubo un verano en donde conocí a dos pasantes con los que compartí y comparto mucho. Me da risa saber que no importa lo que escriba aquí los dos vendrán a molestarme, y por eso los adoro. Ángel, nuestras pláticas sin fin son algo que siempre he atesorado. Me gusta saber que puedo platicar de todo contigo, y más tarde ir a bailar pop o techno (aunque haya que cuidarte después). Tu eres mi niña fresa. Eyal, tal vez te forzamos mucho con la música pop, pero veo que tuvo sus frutos así que no me arrepiento de nada. Gracias por todas las veces que salimos a comer, a compartir un trago y a bailar. Yo se que el bullying es tu lenguaje de amistad, eso aprendí en aquella noche en tu terraza.
De cierta manera ustedes toman el relevo, y aunque se sientan grandes y poderosos siempre estaré aquí para recordarles que los conocí en pañales.

Pierre, muchas gracias por compartir tantas pláticas y chismes sobre todo durante los exámenes. Tú sabes de lo que hablo. Gracias Ferdinand por esas noches de diversion entre música en ruso y música en español. Nina, que llegaras al Labo fue como ver el sol salir después de los largos inviernos en Francia. Eres una persona con tanta vibra positiva y un humor tan como el mio que no puedo evitar de morir de risa contigo.

Gracias a los otros doctorantes, pasados y presentes, del LPSM con los que compartimos pausas, historias de felicidad, de tristeza y de (des)esperanza. Gracias Francesco por siempre traer una sonrisa, Grace por siempre pasar a saludar a mi oficina, Paul por los bares y fiestas sobre todo en Bordeaux. Gracias al trio de grandes Nicklas, Gloria y Thibault con los que pase muchos ratos divertidos alrededor de un café o un vinito, como en Lyon. Te agradezco sobre todo por Gracias Antonio, Yazid, Gabriel, Lucas, Barbara, Mathis, Alessandro, Anna, Arthur, Stan.

Tim amo nuestra amistad que perdura sin importar que pasen años sin vernos, y meses sin hablarnos, porque se que siempre estarás ahí para platicar, reír y ayudarme. 

Romain, te agradezco por los años que compartimos, por todo lo que vivimos y todo lo que maduramos juntos. Nunca olvidaré todos esos bancos que fueron ni todas esas películas y series que devoramos. Llevo conmigo siempre todo lo que fuimos. Espero sigas riendo fuerte y siendo feliz.

El trabajo que concluye hoy empezó de alguna manera en 2015 cuando llegue a Francia, y tengo que hablar de esa persona con la compartimos tantas cosas: Fernanda. Llegamos juntos, y fuimos el soporte del otro durante tanto tiempo, entre llamadas de 5 horas y salidas de fiesta hasta las 6 (con su correspondiente comida antes de dormir). Yo se que no habría logrado quedarme y lograr todo lo que logré sin tu apoyo, pasara lo que pasara, yo sabía que estarías ahí para escucharme. Extraño esos cafés de trabajo con nuestro amigo que siempre andaba escribiendo y leyendo (tu sabes de cual hablo). Me da gusto ver lo que te has vuelto, y aunque ahora estemos tan lejos físicamente, siempre te tengo conmigo. 

Cosette, Cosa, la reina, o como otros te conocen, Carolina. Eres la prueba viviente que la familia se construye, la familia se elige, la familia son esas personas que están ahí siempre. Eres mi familia, eres esa persona que me sostiene, que me da mis regañones, que me escucha, que me mira a los ojos y me dice “Se va a estabilizar”, porque sabes que así será. Tiene tantos años que te conozco que siento que siempre has estado ahí, que siempre has compartido conmigo tu risa, tus chismes, tus consejos de belleza y, sobre todo, tu ser. La vida te trajo tan cerca de mí y soy tan feliz de poder contar contigo. No estaría donde estoy hoy sin ti. Podría escribir mil párrafos agradeciéndote por cada bella cosa que me has dado, por cada acto de afecto y amor, y aunque yo se que amarías, me faltaría papel en Paris para imprimir los manuscritos. Así que resumiré esto en una frase: you are my person.

Gwen. Gracias Gwen. Gracias por existir, gracias por haber aceptado aquella primera cita. Gracias por las risas, los llantos, las palomas, las series, las noches de películas de terror y las mañanas de café viendo Bob’s. Eres de lo más grande que tengo. Eres una persona llena de gentileza, de amor, de humanidad y no dejas de mostrarme tu amabilidad en cada día que llevo conociéndote. Desde el primer día has sido un apoyo incondicional, estuviste ahí en especial durante la escritura de este manuscrito, y vaya que fue difícil, pero gracias a ti pude lograrlo. Gracias por cocinarme, por escucharme, por abrazarme. Gracias por quedarte cuando lo necesito, por darme la libertad que tanto amo. No hay un lugar en donde me sienta más seguro que entre tus brazos. Gracias por ser quien eres. This nest is beautiful because it’s ours.
