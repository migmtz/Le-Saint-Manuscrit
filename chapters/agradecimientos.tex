\chapter{Agradecimientos}

Como lo dicta la tradición tras escribir un tan largo y arduo trabajo, esta sección es una pequeña impresión del agradecimiento que tengo por mucha gente. 
Esta fue la gente que me apoyó durante estos tres años de tesis (y más allá) y sin quienes no habría podido llevar a cabo una tal labor. 

Como muchos saben y otros se estarán dando cuenta, soy una persona que no sabe comenzar las cosas, tampoco terminarlas y mucho menos resumirlas. Sin embargo intentaré con todo mi ser hacer de esta lectura una experiencia amena. Por ejemplo, pueden divertirse contando el número de veces que le agradezco a alguien por los chismes (una de mis actividades preferidas).

Las personas que me conocen sabe que soy una persona que se revitaliza con las relaciones humanas. Muchos me describen como sonriente, lleno de color, de luz, de energía e intensidad. La verdad es que soy lo que soy gracias a la bondad y al amor que todas las personas que menciono aquí tienen por mí.
\vspace{5mm}

Quiero empezar agradeciendo a mis supervisores, quienes tuvieron la difícil tarea de soportarme durante 3 años. Arnaud, gracias por tu carácter determinante, decisivo y tajante, ya sea para arreglar problemas o resolver crucigramas. Maxime y Anna, esta aventura empezó con ustedes cuando decidimos, de una manera vagamente espontánea, trabajar juntos. Anna, gracias por no haber respondido a aquella candidatura que te envié ya que de otra manera nunca me habría aventurado en el mundo de los procesos de Hawkes. Gracias por todos los momentos conviviales, las largas discusiones de matemáticas, ecología y los chismes. Me dio mucho gusto compartir tantas conferencias juntos, en especial las escapadas a Toulouse con los Hawkes. Muchas gracias también por permitirme dar clases con los RESPE y mostrarme que siempre podemos usar nuestro tiempo para ayudar a los demás. Mis años de doctorado habrían sido mucho menos divertidos sin tu buen humor, que me harán tanta falta en lo que viene. Maxime, gracias por hacer de tu bureau nuestro cuartel general de reuniones, de siempre sacar las galletas ocultas cuando nos veíamos a las tempranas 9 de la madrugada y sobre todo por tu rigor científico. Aunque no tengamos las mismas maneras de trabajar (un día te convenceré de venir a trabajar a un café conmigo) me da gusto saber que contigo siempre es posible reír y bromear. Les agradezco porque siento que hacer un doctorado es un reto enorme y estoy contento de lo que logramos hacer juntos.

Le agradezco a mi jurado por haber aceptado evaluar mi trabajo. Gracias François Roueff y Gordon J. Ross por haber tomado el tiempo de leer mi manuscrito. Gracias Céline Duval y Sophie Donnet por haber aceptado ser mis examinadoras.

Muchas gracias tambien a Eva Löcherbach, Viet Chi Tran y Etienne Roquain por haber formado parte de mi comité de seguimiento de tesis.

Felix, te agradezco por haber sido un guía de los procesos de Hawkes y sobre todo de la teoría espectral. Recuerdo haberte conocido en tu posdoctorado en el laboratorio y como toda buena historia ahora continuamos a trabajar juntos esta vez para el mio. Gracias por siempre dar de tu tiempo para explicarme cosas, para compartir con tanta pasión tus hobbies, opiniones y conocimientos, o simplemente para charlar.
Aprovecho para agradecerle a la comunidad de Hawkes con los que compartí conferencias, seminarios, pláticas y comidas, ya sea en Toulouse, Lille, Paris o donde sea (porque al parecer estamos en todos lados).

Le agradezco a todos los miembros del LPSM que animan la vida científica de esta universidad con sus personalidades tan variadas, tan únicas pero sobre todo siempre bondadosas. Gracias Claire por haberme integrado en este laboratorio desde mi época de pasante y por hacernos sentir a mi y a los otros doctorantes como verdaderos colegas. Suerte en tu nueva aventura. Gracias Stéphane por siempre tener una sonrisa, por siempre echar la plática y compartir tu vasto saber. Espero que estos años como vecinos de oficina donde te toco escuchar música, risas y cantos hayan sido no muy desagradables. Gracias Antoine por siempre recordarnos a los doctorantes que hay que disfrutar de la vida y saber cuando dejar de trabajar. Gracias Charlotte por tu gentileza, por organizar el grupo de Hawkes. Fue muy agradable formar parte del grupo de profesores de L3 Estadística contigo. Gracias a los demás miembros Anna BH, Jean-Patrick, Maud, Erwan, Sylvain, Lorenzo (y todos y todas que habré olvidado poner aquí). 
Le agradezco de la misma manera a los miembros de la administración Valerie,  Nathalie, Marouane. En particular, gracias Louise que me acompañaste de pasante y en mis primeros años de tesis. Adoré cada momento que pude platicar contigo, mil gracias por constantemente preguntar como podías ayudar, hiciste que mi vida laboral fuera más amena. Gracias Hugues, el Atlas que carga el mundo informático de este laboratorio a base de mails con un humor singular e hilarante. No hay nada más divertido que leer que no puedes hacer algo “inmediatamente” pero que en 1 hora estará listo. Si hubiera sido tan eficaz, habría probablemente publicado 77 papeles durante mi tesis. Gracias a todos los miembros de limpieza porque gracias a ustedes podemos trabajar en condiciones óptimas.

\vspace{5mm}

En toda aventura se conoce gente que nos acompaña y con quienes compartimos logros, felicidad, desesperación y sobre todo cafés… muchos cafés. Camila, Cami, no podría estar más feliz de haber estado en la misma oficina contigo, nada mejor que tener un pedacito de Latinoamérica tan cerca. Aprecio cada momento de canto y música, de enviarnos memes, de pegar el rostro del otro en las sillas y más que nada la felicidad de gritar cuando nos volvemos a ver, no importa que hayan pasado unos días o varias semanas sin vernos. Aún recuerdo perfectamente que tan cercanos nos volvimos en Porquerolles y nuestro subsecuente verano bailando cada viernes en la noche en los quais. Que sepas que admiro muchísimo la intensidad y la determinación que te lleva a hacer todo lo que te gusta con tanto fervor. Espero seguir con nuestras sesiones de shopping, karaoke, chisme y comida por años y años, porque contigo la vida es tan divertida y amena. Te quiero más que a la diosa Shaki.
Iqraa, gracias por esas pláticas, a veces tan filosóficas, en el quinto piso o en un café perdido de Paris. Eres una persona con mucho corazón y que sabe ser fiel a sus principios y creencias: que sepas que es algo que admiro muchísimo de ti. Gracias porque contigo siempre puedo contar para bailar y cantar a todo pulmón. Gracias por haber compartido un Airbnb conmigo en nuestra conferencia en Lyon, donde nos quedabamos hasta tarde platicando, fue muy agradable vivir contigo esos días. Sobre todo, gracias por todas esas veces que me escuchaste y me tendiste los brazos para abrazarme. 
Ariane, aprecio mucho el habernos acercado tanto con los años y formar parte de ese grupo exclusivo de gente que puede abrazarte (con moderación). En particular, recuerdo mucho aquella vez que terminamos comiendo por la noche en Burger King, o aquella plática llena de emociones en Bruselas sentados en la barra de un restaurante japonés. 
Entre mis mejores recuerdos con todas ustedes están esas pláticas de chisme, amores y desamores, risas y llantos. 
Ahora que oficialmente nos vamos todos del Labo, habrá que encontrar donde hacer nuestras sesiones de películas en verano !
Quiero que sepan que su amistad es un regalo que atesoro, y que su amor y cariño llena de color y vida mi día a día.

Ahora toca agradecerle a dos personas que conocí en la Maestría. Claire, Clairounette, Clairicot, ma chère Claire, soy tan pero tan feliz de haberte conocido, de haberte propuesto de pasar tu pasantía en mi oficina porque fue el inicio de una amistad que hasta hoy es de las más preciadas para mí. Tienes más energía que el Sol, siempre te veo corriendo de derecha a izquierda (a veces literalmente), ya sea para compartir unos juegos de mesa en invierno o la playa en verano (mil mil mil gracias por hacerme descubrir Royan). Llevo siempre conmigo todas esas largas y largas caminatas por todo París, compartiendo ideas, opiniones, chismes y comida, mucha mucha comida. Gracias también por todos los consejos y por siempre ser tan directa conmigo. No importa donde estés, siempre estarás cerca mio. Te mando mis mejores deseos para este nuevo episodio que estás comenzando.
Ludovic, espero que andes leyendo la versión en español de los agradecimientos. Ludo, amo verte sobrellevar la vida con una sonrisa, con calma, con una aura tan chill. Siempre has sabido bromear conmigo y por siempre compartir tu pasión por la cultura, la historia, y los bebés feos del arte sacro. En especial, gracias por bailar conmigo en tantas noches al borde del río (no te preocupes ya no me duele el codo).

Gracias Alexis. Gracias por todas las noches comiendo tu deliciosa comida. Gracias por enseñarme el mundo de las fiestas techno, por haberme acompañado a tantas fiestas pop (por más que decías que las odiabas). Te quiero agradecer por todo el cuidado que me diste, por todo ese amor y amistad, por todos esos tragos que compartimos, esas pláticas, los tantos chismes, los grafos, los consejos, los abrazos. Acercarme a ti fue de las cosas mas bonitas que me sucedieron en el doctorado. Sigue viviendo con la pasión e intensidad con la que la techno te hace bailar.
Antonin, mi ciela, que gusto me da tenerte en mi vida, poder reírnos tanto con esos edits de las estrellas pop, el poder cantar a todo pulmón Beyoncé y que siempre estés ahí para escucharme.
Sin importar lo que pase, ustedes siempre ocupan un lugar especial en mi corazón y espero poder tenerlos cerca por años.

Le agradezco a una doctorante que es tan dedicada como feminista e hiperactiva: Adeline. No pensé encontrar alguien que piensa y habla tan rápido como yo, nuestras platicas en mi oficina (o la del jefe) eran caóticas, divertidas y llenas de consejos amorosos. Muchas gracias porque supiste venir y darme apoyo cuando lo necesité tanto. Aún recuerdo ver Batman juntos aquella noche.

Hubo un verano en el que conocí a dos pasantes con los que compartí y comparto mucho. Me da risa saber que no importa lo que escriba aquí los dos vendrán a molestarme, y por eso los adoro. Ángel, nuestras pláticas sin fin son algo que siempre he atesorado. Me gusta saber que puedo platicar de todo contigo, y más tarde ir a bailar pop o techno (aunque haya que cuidarte después). Tu eres mi niña fresa. Eyal, tal vez te forzamos mucho con la música pop, pero veo que tuvo sus frutos así que no me arrepiento de nada. Gracias por todas las veces que salimos a comer, a compartir un trago y a bailar. Yo se que el bullying es tu lenguaje de amistad, eso aprendí en aquella noche en tu terraza. Yo se que puedo contar contigo y que sepas que yo también estoy aquí para ti.
De cierta manera ustedes toman el relevo, y aunque se sientan grandes y poderosos siempre estaré aquí para recordarles que los conocí en pañales.

Pierre, muchas gracias por divertirme tanto durante los exámenes (tú sabes de lo que hablo). Gracias Ferdinand por esas noches de diversion entre música en ruso y música en español. Nina, que llegaras al labo fue como ver el sol salir después de los largos inviernos en Francia. Eres una persona con tanta vibra positiva y sassiness que no puedo más que adorar tu compañía.

Gracias a los otros doctorantes pasados y presentes del LPSM con los que compartimos pausas, historias de felicidad, de tristeza y de (des)esperanza. Gracias Francesco por siempre traer una sonrisa, Grace por siempre pasar a saludar a mi oficina, Paul por los bares y fiestas sobre todo en Bordeaux, Romain por las pláticas de Hawkes. Gracias al trio de grandes Nicklas, Gloria y Thibault con los que pase muchos ratos divertidos alrededor de un café o un vinito, como en Lyon. Gracias Antonio, Yazid, Gabriel, Lucas, Barbara, Mathis, Alessandro, Anna, Arthur, Stan.

Haydee, beibi, muchas gracias por ser quien eres, por reír a todo pulmón conmigo. Atesoro tanto los recuerdos tan largos, numerosos y diversos que tenemos. Después de 11 años de amistad, me parece una locura ver que cada que nos vemos seguimos tan conectados. No puedo creer que seguimos a veces terminando la frase o las bromas del otro. Tu amistad vale oro puro para mí y te agradezco tanto pero tanto por permitirme tenerte tan cerca. Te adoro.

Tim, amo nuestra amistad que perdura sin importar que pasen años sin vernos, y meses sin hablarnos, porque se que siempre estarás ahí para platicar, reír y ayudarme. Que locura que ya llevemos 8 años conociéndonos. No puedo evitar reír con todos los recuerdos que tenemos en Toulouse 

Romain, te agradezco por los años que compartimos, por todo lo que vivimos y todo lo que maduramos juntos. Nunca olvidaré todos esos bancos que fueron ni todas esas películas y series que devoramos. Llevo conmigo siempre todo lo que fuimos, gracias por acompañarme al inicio de este viaje. Espero sigas riendo fuerte y siendo feliz.

El trabajo que concluye hoy empezó de alguna manera en 2015 cuando llegue a Francia, y tengo que hablar de esa persona con la que empezó todo: Fernanda. Llegamos juntos, y fuimos el soporte del otro durante tanto tiempo, entre llamadas de 5 horas y salidas de fiesta hasta las 6 (con su correspondiente comida antes de dormir). Yo se que no habría logrado quedarme y lograr todo lo que logré sin tu apoyo ya que, pasara lo que pasara, yo sabía que estarías ahí para escucharme. Extraño esos cafés de trabajo con nuestro “amigo” que siempre andaba escribiendo y leyendo. Me da gusto ver lo que te has vuelto, y aunque ahora estemos tan lejos físicamente, siempre te tengo conmigo. 

Alfredo, me da tanto gusto el rencontrarnos y darnos cuenta que desde el liceo pensábamos en acercarnos. Estoy feliz por todo lo que has logrado, y como sigues evolucionando y avanzando. Sigamos cantando, bailando y criticando mi cochina.

\vspace{5mm}
Es momento de agradecerle a la persona más importante en este mundo para mí. Mi señora madre, mamá, hermosa. Muchas gracias por haberme traído, por haberme criado, por haberme hecho la persona que soy hoy, fuerte, dedicada y sobre todo feliz. Es gracias a ti que puedo ver mis manos llenas de frutos de todo lo que he sembrado. Espero poder darte todo lo que te mereces por todo el amor que me has dado y los sacrificios que has hecho. Todo lo que soy, lo que tengo, lo que deseo es por ti, vive por ti, respira por ti. Tenerte tan lejos es extrañarte a un punto que parece a veces insoportable pero yo se que siempre estarás aquí conmigo, y que sepas que siempre estoy ahí para ti. Te amo con todo lo que soy.

Lidi, muchas gracias por haber sido como mi segunda madre, gracias porque tu me hiciste siempre sentir parte de tu familia, me diste tanto amor que a veces no se como retribuirlo. Fuiste la que primero me mostró el mundo de este lado del charco. Siempre estuviste para apapacharme, consentirme y apoyarme en tantas pero tantas cosas. Te amo idi.

Fa, papá, muchas gracias por todo lo que me has dado, por la paciencia y el amor que me has tenido. Algo que siempre me dio y me da seguridad es saber que siempre estás ahí para mí. Que siempre estarás cuando lo necesito y que por más que nuestros humores choquen (lo heredé un poquito de ti hahaha) siempre me amas con todo tu ser. Te amo.

Muchas gracias Mar porque crecer contigo fue como tener una hermana que me protegía y cuidaba. No sabes como te agradezco porque siempre te sentí tan gentil conmigo. Gracias a ti entendí muchas cosas y mi vida no habría sido para nada lo que es hoy sin ti. Marco, German, gracias por haberme tratado como a un hermano (con todo lo bueno y malo que viene en ese paquete) porque gracias a ustedes me sentí mucho menos solo.

Mari, prima, Mariana, tu alma, tu presencia, tu ser me hacen sentir acompañado en este océano gigantesco que es la realidad. Pensamos, sentimos, vivimos la vida de una manera tan similar, que seguramente fue lo que nos acerco desde chicos y nos unió tan rápido y tan fuerte. Gracias por cada segundo donde me apoyaste, donde me escuchaste y me diste tus consejos. Te quiero tanto pero tanto que quisiera tener tu pluma para poder expresarlo con palabras. Sigue tu fuego interior siempre, nunca olvides quien eres, lo que has logrado, y el poder que tienes, porque al menos mi vida la cambiaste radicalmente. 

Nay, Ive, agradezco tanto a la vida por habernos acercado tanto cuando lo hizo. Soy tan feliz que sean mis hermanas y que me muestren su amor en cada oportunidad que tienen. Bailar con ustedes, cantar con ustedes, festejar con ustedes siempre ha tenido un calor singular que me llena de felicidad. Las adoro.

Palola, Pola, Dude, mi infancia no habría sido infancia sin ti, y mis pulmones no serían tan fuertes sin las horas y horas de risas que compartimos. No creo conocer a nadie que me haya dado tantos dolores de estomago con tanta tontería. Tengo tantos recuerdos de pelearnos, encontentarnos, hacer travesuras, hacer enojar a nuestras madres y reír a nuestros abuelos. Te extraño mucho y te quiero todavía más.  

Quiero agradecerle a mis tías porque, como muchos notaran, es gracias a mujeres muy fuertes que soy lo que soy. Muchas gracias Lale por tus cuidados y por todas las tardes de películas, telenovelas y sobre todo risas con Pao y mi mamá. Maruka, extraño tanto las cositas de la mujer en las noches tu y yo. Tengo mis recuerdos llenos de pequeños momentos contigo que tal vez no he sabido decirte a que punto llenan mi alma de dulzura, calidez y tranquilidad. Gracias. Gracias Kenuki y Elsuki porque me hicieron siempre sentir como un niño, algo que adoro. Extraño tanto las navidades con ustedes en las que nos la pasábamos en la cocina entre locuras, gritos y, mis favoritos, chismes. Gracias Ceci por siempre haber estado ahí y por compartir tu amor (y tu amor por el pastel de fresa hahah).

Quiero agradecerle a mis abuelitos y abuelitas que siempre me llenaron de amor y abrazos, cada uno y una a su manera. Espero que estén donde estén se sientan bien y orgullosos.


\vspace{5mm}
Para terminar con estos agradecimientos, quiero hablar de dos personas que son mis mas grandes tesoros.

Cosette, Cosa, la reina, o como otros te conocen, Carolina. Eres la prueba viviente que la familia se construye, la familia se elige, la familia son esas personas que están ahí siempre. Eres mi familia, eres esa persona que me sostiene, que me da mis regañones, que me escucha, que me mira a los ojos y me dice “Se va a estabilizar”, porque sabes que así será. Tiene tantos años que te conozco que siento que siempre has estado ahí, que siempre has compartido conmigo tu risa, tus chismes, tus consejos de belleza y, sobre todo, tu ser. La vida te trajo tan cerca de mí, y soy tan feliz de poder contar contigo. No estaría donde estoy hoy sin ti. Podría escribir mil párrafos agradeciéndote por cada bella cosa que me has dado, por cada acto de afecto y amor, y aunque yo se que amarías, me faltaría papel en Paris para imprimir los manuscritos. Así que resumiré esto en una frase: you are my person.

Gwen. Gracias Gwen. Gracias por existir. Gracias por haber aceptado aquella primera cita. Gracias por las risas, los llantos, las palomas, las series, las noches de películas de terror y las mañanas de café viendo Bob’s. Eres de lo más grande que tengo. Eres una persona llena de gentileza, de amor, de humanidad y no dejas de mostrarme tu amabilidad en cada día que llevo conociéndote. Desde el primer día has sido un apoyo incondicional, estuviste ahí en especial durante la escritura de este manuscrito, y vaya que fue difícil, pero gracias a ti pude lograrlo. Gracias por cocinarme, por escucharme, por abrazarme. Gracias por quedarte cuando lo necesito, por darme la libertad que tanto amo. No hay un lugar en donde me sienta más seguro que entre tus brazos. Gracias por ser quien eres. This nest is beautiful because it’s ours.
