\chapter{Remerciements}

Vous allez découvrir ici que deux choses que Miguel ne sait pas faire est de 
commencer et de finir les choses. Déjà, choisir une langue dans laquelle écrire ces 
remerciements a été une épreuve monumentale. Alors je décide de les faire en français,
non pas par "malinchismo" mais plutôt car c'est la langue que je pense fera le plus
grand consensus parmi les lecteurs de ces paragraphes.

Je vais faire de mon mieux pour être succinct mais je pense que la plupart d'entre
vous me connaissez déjà et vous savez que ceci risque d'être long, parce que je suis 
quelqu'un qui adore exprimer tous ses bons sentiments, et de bons sentiments il 
y en aura beaucoup. 

Sachez que l'ordre de ces remerciements ne sont pas à "overthink", mon conseil pour
vous mes lecteurs dont les prénoms apparaîtront ici est de le voir comme aller 
au parc d'attractions : l'important pour vous n'est pas dans quel ordre vous passez 
mais plutôt de profiter de la montagne russe une fois que c'est votre tour.

\vspace{1cm}

Je vais commencer par remercier mes encadrants avec qui je me suis lancé dans cette 
aventure et sans qui je n'aurai pas réussi à produire à la fois autant de sciences 
comme des rires. Arnaud, tu as toujours été là, comme on dit au Mexique, au pied du 
canon. Tu m'as offert ton aide dès que tu avais la possibilité et moi le besoin. 
Tes cours de M1 tout comme ta gentillesse et ton grand niveau de culture ne sont pas 
des choses que je risque d'oublier. 
Anna et Maxime, la dream team de Hawkes, je me souviens d'avoir commencé ce stage 
avec vous à la fin de mon Master sans savoir trop quoi faire de ma vie. J'ai commencé 
convaincu de pas vouloir faire une thèse, et après ces moments d'hésitation 
(très caractéristiques de moi), me voir changer radicalement d'avis. J'ai dû attendre 
un an pour continuer ce projet sur nos chers processus de Hawkes, tout simplement 
car je voulais vous avoir comme encadrants et je sais que c'est un choix que je ne 
regretterai jamais. J'ai voulu rester avec vous car on s'entend bien, car on forme 
une équipe que je pense les gens de l'extérieur se demandent comment cela peut marcher 
mais mon dieu que cela marche bien. J'adore pouvoir toujours jongler entre le sérieux 
et la taquinerie sans jamais perdre ni le respect ni la complicité. 

Maxime, je suis 
heureux d'avoir passé ces 4 ans et demie ensemble car j'ai pu voir tout ce qui se 
cachait derrière ce visage stoïque que je voyais en TD aux débuts de ma vie à Paris. 
Tu as toujours su offrir un sourire tout en nous demandant de ralentir quand Anna et 
moi on s'emportait sur nos discussions Hawkesiennes. Sans tes feuilles remplies de 
corrections écrites avec le rouge le plus sanglant je n'aurai jamais pu écrire des 
preuves aussi propres, que je suis fier de montrer au monde.

Anna, je suis content que tu aies 
oublié de me répondre pour l'offre de thèse au début, car cela m'a permis de voir 
que vraiment parfois les gens sont destinés de se rencontrer même dans le milieu 
professionnel. Je chéris tous ces moments passés ensemble à parler pendant des heures. 
Je te remercie de m'avoir motivé à travailler pour le DU RESPE qui est une des 
expériences d'enseignement les plus belles et enrichissantes que j'ai eu. Je 
garde toujours près de moi aussi nos moments ensemble en conférence qui n'ont fait 
que me permettre de mieux te connaître.

Je profite ici pour remercier toute ma chère communauté Hawkes que j'ai rencontré 
au fil des années et que j'espère continuer de croiser, d'écouter et de travailler 
avec. En particulier, je veux remercier Felix pour ces instants passionnants de 
discussions autour du spectral, j'espère on va réussir à lancer un mouvement (et 
comprendre un jour l'entièreté de Daley et Vere-Jones).

REMERCIER JURY

\vspace{1cm}

Je veux parler des personnes dans mon laboratoire dont le travail m'a permis de 
travailler dans les bonnes conditions, et que parfois l'on ne remercie pas assez. 
Merci au secrétariat, en particulier à 
Nathalie, Valérie, Hugues, car c'est grâce à vous que beaucoup de choses marchent, 
c'est grâce à vous que j'ai une chaise où m'asseoir, un ordi sur lequel travailler,
des billets de train pour aller en conf et beaucoup plus. Je veux remercier le 
personnel de nettoyage qui ont toujours su avoir un sourire et un chaleureux "Bonjour"
chaque matin. 

Et je donne ici un paragraphe spéciale pour Louise, que j'ai rencontré aux débuts de 
mon stage et qui vient toujours nous voir de temps en temps. Louise, ta bonne humeur, 
ton sourire, tes blagues, ton efficacité sont sans égaux. Je suis content de t'avoir 
croisé, d'avoir pu discuter avec toi et surtout de voir comment tu sais profiter de 
la vie.

\vspace{1cm}

Passons au LPSM et toutes ces personnes que j'ai pu croiser et qui ont fait de ces 
4 ans une expérience inoubliable. 
