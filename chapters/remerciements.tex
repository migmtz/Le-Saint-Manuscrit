\chapter{Remerciements}

Comme le veut la tradition après avoir réalisé un aussi long et éprouvant travail, cette section est une petite impression de la gratitude que j’ai pour beaucoup de gens.
Ici se trouvent les gens qui m’ont soutenu pendant ces trois dernières années (et bien au-delà) et sans qui je n’aurai pu mener à bout un tel travail.

Comme beaucoup le savent, et d’autres le découvrent, je suis une personne qui ne sait pas commencer les choses, ni les conclure, et encore moins les résumer. Toutefois, je vais essayer avec tout mon être de faire de cette lecture une expérience chaleureuse, ou du moins amusante. Par exemple, vous pouvez vous amuser à compter le nombre de fois que je remercie quelqu'un pour les ragots (une de mes activités préférées). 

Les personnes qui me connaissent savent que je suis une personne qui se revitalise à travers les relations humaines. Beaucoup me décrivent comme souriant, plein de couleurs, de lumière, d’énergie et d’intensité. Il s’avère que je suis tout cela grâce à la bienveillance et l’amour que toutes les personnes mentionnées ici éprouvent pour moi.
\vspace{5mm}

Je veux commencer en remerciant mes encadrants qui ont eu la lourde tâche de me supporter pendant 3 ans. Arnaud, merci pour ton caractère déterminé, décisif et tranchant, que ce soit pour résoudre des problèmes ou des mots-croisés. Maxime et Anna, cette aventure a commencé avec vous quand nous avons décidé, d’une manière vaguement spontanée, de travailler ensemble. Anna, merci de ne pas avoir répondu à cette candidature que je t’avais envoyé car autrement je ne me serais pas lancé dans le monde des processus de Hawkes. Merci pour tous ces moments de convivialité marqués par tant de discussions de maths, d’écologie et de ragots. Je suis tellement ravi d’avoir partagé autant de conférences ensemble, en particulier nos escapades à Toulouse avec les Hawkes. Merci également de m’avoir permis d’enseigner chez les RESPE et de m’avoir montré que nous pouvons toujours utiliser notre temps pour aider les autres. Mes années de doctorat auraient été beaucoup moins amusantes sans ta bonne humeur qui me manquera tant à l’avenir. Maxime, merci d’avoir fait de ton bureau notre quartier général, d’avoir toujours sorti les cookies secrets quand on se voyait à 9 heures de l’aube, et surtout de ta rigueur scientifique. Même si nos manières de travailler ne sont pas toujours alignées (un jour je te convaincrai de venir travailler dans un café avec moi), je suis content de savoir que je pouvais toujours rire et blaguer avec toi. Je vous remercie car faire un doctorat est un défi énorme et je suis content de ce que l’on a accompli ensemble.

Merci à mon jury d’avoir accepté d’évaluer mon travail. Merci François Roueff d’avoir pris le temps de lire mon manuscrit. Thank you Gordon J. Ross for having taken the time to read my manuscript. Merci Céline Duval et Sophie Donnet d’avoir accepté d’être mes examinatrices. 

Merci beaucoup aussi à Eva Löcherbach, Viet Chi Tran et Etienne Roquain d'avoir accepté d'être dans mon comité de suivi de thèse.

Felix, je te remercie car tu as été un guide dans l’univers des processus de Hawkes et surtout de la théorie spectrale. Je me souviens de t’avoir rencontré dans tes années de postdoc au laboratoire, et comme toutes les bonnes histoires, on continuera de travailler ensemble cette fois pour le mien. Merci de toujours avoir donné ton temps pour m’expliquer des choses, pour discuter et pour partager avec autant de passion tes hobbies, tes opinions et connaissances. 
Je profite ici pour remercier aussi à la communauté Hawkes avec qui j’ai pu partager de nombreuses conférences, séminaires, discussions et repas, que ce soit à Toulouse, Lille, Paris ou ailleurs (car au final on est un peu partout).

Je remercie tous les membres du LPSM qui animent la vie scientifique de cette université avec vos personnalités aussi variées qu’uniques et bienveillantes. Merci Claire de m’avoir intégré dans la vie de ce labo depuis mon époque en tant que stagiaire, et de nous avoir fait sentir, moi et les autres doctorants, comme de vrais collègues. Bon courage dans ta nouvelle aventure. Merci Stéphane de toujours avoir un sourire, de toujours partager une bonne discussion ou tes connaissances. J’espère que ces années où nous avons été voisins, où tu as eu affaire à de la musique, aux rires et aux chants, n’auront pas été trop désagréables. Merci Antoine de toujours être là pour rappeler aux doctorants que parfois il faut arrêter de bosser et profiter de la vie dehors. Merci Charlotte de ta gentillesse et d’avoir organisé le séminaire de Hawkes. J’ai beaucoup aimé faire partie du groupe de TD de la L3 Statistiques avec toi. Merci aux autres membres: Anna BH, Jean-Patrick, Maud, Erwan, Sylvain, Lorenzo (et celles et ceux que j’aurai oublié ici).

Merci aux membres de l’administration du labo, Valérie, Nathalie, Marouane. En particulier, merci Louise de m’avoir accompagné dans mon stage et dans mes premières années de thèse. J’ai adoré chaque moment où nous avons papoté, merci infiniment de toujours me demander comment tu pouvais aider car tu rendait ma vie au travail tellement plus agréable. Merci Hugues, Atlas portant le monde informatique de ce labo à base de mails avec ton humour hilarant et singulier. Il n’y a rien de plus drôle que de lire que tu ne pouvais pas faire un truc “dans l’immédiat” mais que dans une heure tout serait prêt. Si j’avais été aussi efficace que toi j’aurais sûrement publié 77 papiers de plus. Merci aussi aux membres du ménage qui nous permettent de travailler dans les meilleures conditions chaque jour.

\vspace{5mm}
Dans toute aventure on rencontre des gens qui nous accompagnent et avec qui nous partageons nos réussites, notre bonheur, notre désespoir et, surtout, du café… beaucoup de café. Camila, Cami, je ne peux pas expliquer à quel point je suis le plus heureux d’avoir partagé le bureau avec toi; rien de mieux que d’avoir un petit morceau de l’Amérique Latine aussi proche. Je chéris chaque moment de chant et de musique, de s’envoyer des memes, de coller le visage de l’autre dans nos chaises et, en particulier, de crier de joie à chaque rencontre, peu importe si c’était après quelques heures ou plusieurs semaines sans se voir. Je me souviens parfaitement de à quel point on est devenus proches à Porquerolles et l'été d'après où l'on a passé à danser ensemble chaque vendredi soir aux quais. J’admire tellement ton intensité et ta détermination à faire tout ce qui te plaît en t’investissant à fond. J’espère que l’on continuera nos sessions de shopping, de karaoke, de ragot et de nourriture pendant longtemps car la vie avec toi est tellement amusante et agréable. Je t’aime plus qu’à notre déesse Shaki.
Iqraa, merci pour toutes ces discussions, parfois très philosophiques, au cinquième étage ou dans un café perdu dans Paris. Tu es une personne avec un coeur énorme et qui sait être fidèle à ses principes et à ses croyances: que tu saches que c’est une chose que j’apprécie énormément chez toi. Merci, car je peux compter sur toi pour danser et chanter jusqu’à l’épuisement. Merci d'avoir partagé un Airbnb avec moi pendant la conférence à Lyon où l'on restait jusqu'à tard le soir à discuter, j'ai adoré vivre avec toi ce petit bout de temps. En particulier, merci pour toutes ces fois où tu m’as écouté et tu m’as tendu les bras pour me câliner. 
Ariane, je suis content de nous être approchés autant avec les années et de savoir que je fais partie des chanceux qui peuvent te câliner (avec modération). Je pense en particulier à cette nuit qu'on a fini par manger Burger King, et à cette discussion plein d'émotions à Bruxelles assis au bar d'un restaurant japonais.
Parmi mes meilleurs souvenirs avec vous toutes il y a tous ces ragots, toutes ces discussions d’amour, de désamour, tous ces rires et pleurs..
Maintenant que nous sommes tous et toutes partis du labo il faudra bien que nous retrouvions où continuer les films de l’été !
Je veux que vous sachiez que votre amitié est un cadeau que je chéris, et que votre amour et affect remplissent de couleur et de vie mon quotidien.

Maintenant le spotlight se tourne vers deux personnes que j’ai rencontrées dans le Master. Claire, Clairounette, Claricot, ma chère Claire, je suis tellement mais tellement heureux de t’avoir rencontré, de t’avoir proposé de faire ton stage dans mon bureau car ça a marqué le début d’une amitié qui, aujourd’hui, est l’une de plus chères pour moi. Tu as plus d’énergie que le Soleil même, je te vois courir de droite à gauche (parfois littéralement), que ce soit pour partager un jeu de société en hiver ou la plage en été (merci beaucoup beaucoup beaucoup de m’avoir fait découvrir Royan). Je porte dans moi toutes ces longues marches ensemble dans Paris à partager des idées, des opinions, des ragots et de la nourriture… beaucoup de nourriture. Je te remercie aussi pour tous tes conseils et de toujours être aussi directe avec moi. Peu importe où tu seras, tu seras toujours à mes côtes. Je t’envoie mes meilleurs souhaits dans ce nouvel épisode qui démarre pour toi.
Ludovic, j’espère que tu es en train de lire ceci mes remerciements en espagnol. Ludo, j’adore voir ta manière de vivre la vie avec un sourire, avec un calme propre à toi, avec un aura tellement chill. Tu as toujours su rigoler avec moi et partager ta passion pour la culture, l’histoire et les bébés moches de l’art sacré. Surtout, merci d’avoir dansé avec moi autant de nuits dans les quais (et je te rassure, je n’ai plus mal au coude).

Merci Alexis. Merci pour ta nourriture aussi délicieuse, Merci pour toutes les soirées partagées chez toi, en soirée techno, ou même dans les soirées pop (auxquelles tu m’accompagnais même si tu disais les détester). Je veux te remercier d’avoir pris soin de moi, pour tout l’amour et l’amitié que tu m’as donné, pour tous les verres, les discussion, les ragots, les graphes, les conseils, les câlins. Te rencontrer est l’une des meilleures choses qui me sont arrivées pendant le doctorat. Continue de vivre avec la même intensité et passion avec laquelle la techno te fait danser.
Antonin, mi ciela, je suis heureux de t’avoir dans ma vie, de rire autour de ces edits des stars pop, de pouvoir chanter Beyoncé de toutes nos forces. Merci d’être toujours là pour moi.
Peu importe ce qui se passe, vous occupez un endroit spéciale dans mon coeur et j’espère pouvoir vous garder proche de moi pour des années à venir.

Je remercie une doctorante aussi dédiée que féministe et hyperactive: Adeline. Je ne pensais pas rencontrer quelqu’un qui pense et qui parle aussi vite que moi, nos discussions dans mon bureau (ou celui du chef) étaient chaotiques, drôles et remplies de conseils d’amour. Merci beaucoup car tu as su venir et me donner ton support quand j’en avais beaucoup besoin. Je me souviens très bien de notre soirée à regarder Batman.

Il y a eu un été où j’ai rencontré deux stagiaires avec qui j’ai beaucoup partagé, et partage encore aujourd’hui. Ça me fait rire de savoir que, peu importe ce que j’écrirai ici, ils viendront tous les deux m’embêter, et pour cela je vous adore. Ángel, nos discussions sans fin sont un trésor pour moi. J’aime savoir que je peux discuter de tout et de rien avec toi pour plus tard aller danser autour de la pop ou de la techno (même s’il faut s’occuper de toi après). Tu es ma niña fresa. Eyal, on t’a peut-être beaucoup imposé la musique pop mais je vois que cela a fait ses fruits donc je ne regrette rien. Merci pour toutes les fois où nous sommes allés manger, prendre un verre et danser. Je sais que le bullying est ta façon d’éprouver ton amitié, je l’ai bien appris à cette soirée sur ta terrasse. Je sais que je peux compter sur toi et je veux que tu saches que c’est réciproque.
D’une certaine façon vous prenez le relais, et bien que vous vous sentiez être grands et puissant je serai toujours là pour vous rappeler que vous portiez des couches quand je vous ai rencontrés.

Pierre, merci beaucoup de m’avoir fait autant rire pendant les surveillances (tu sais de quoi je parle). Merci Ferdinand pour les nuits entre musique en russe et musique en espagnol. Nina, ton arrivée au labo a été comme voir le soleil sortir après les longs hivers en France. Tu es une personne avec une énergie tellement positive et sassiness que je ne peux pas m’empêcher d’adorer ta compagnie.

Merci à tous les autres doctorants passés et présents du LPSM avec qui on a passé tellement de pauses ensemble, tellement d’histoires de bonheur, de tristesse et de (des)espoir. Merci Francesco de toujours avoir un sourire, Grace de toujours venir venir faire coucou dans mon bureau, Paul pour les bars et les soirées surtout à Bordeaux, Romain pour les discussions de Hawkes. Merci au trio mythique de Nicklas, Gloria et Thibault avec qui je me suis amusé à de nombreuses reprises autour de cafés, bières et vins fancy, comme à Lyon. Merci Antonio, Yazid, Gabriel, Lucas, Barbara, Mathis, Alessandro, Anna, Arthur, Stan.

Haydee, beibi, merci beaucoup d’être la personne que tu es et de toujours rire de toutes tes forces avec moi. Je chéris énormément ces nombreux souvenirs aussi divers que l’on partage. Après 11 ans d’amitié, cela me paraît fou de voir à quel point on est toujours aussi connectés, à finir les phrases et les blagues de l’autre. Ton amitié c’est de l’or pur pour moi et je te remercie tellement de me laisser être aussi proche de toi. Je t’adore.

Tim, j’adore notre amitié qui persiste au fil des années, peu importe le temps que nous passons sans se voir. Je sais que tu seras toujours là pour discuter, rire et m’aider. C’est déjà 8 ans de se connaître et je rigole encore en pensant à tout ce que l’on a partagé à Toulouse.

Romain, je te remercie pour toutes ces années ensemble, tout ce que nous avons vécu et pour toute la maturité que l’on a acquise ensemble. Je n’oublierai jamais tous ces bancs qui ont été ni tous ces films et séries que nous avons dévorés. Je porte avec moi tout ce que nous avons été, et merci de m’avoir accompagné dans le début de ce voyage. J’espère que tu continues de rire fort et d’être heureux.

Le travail qui conclut aujourd’hui a commencé en quelque sorte en 2015 quand je suis arrivé en France et je dois parler d’une personne avec qui tout a commencé : Fernanda. Nous sommes arrivés ensemble et on a été le pilier l’un pour l’autre pendant tellement longtemps, entre appels de 5 heures et sorties jusqu’à 6h du matin (avec son respectif repas avant de dormir). Je sais que sans toi je ne serais pas resté ici et je n’aurai pas accompli autant, car tu étais toujours là pour m’écouter. Ça me manque tous ces cafés à travailler à côté de notre “pote” de café qui passait à lire et à écrire. Je suis content de voir ce que tu es devenue, et que tu saches que même si tu es loin physiquement, je te sens toujours proche de moi.

Alfredo, je suis vraiment content de s’être retrouvés et de s’être rendu compte que nous nous cherchions autant depuis le lycée. Je suis heureux de tes réussites et surtout de voir comment tu continues d’évoluer et d’aller vers l’avant. Continuons de danser, chanter et critiquer mi cochina.

\vspace{5mm}
C’est le moment de remercier la personne la plus importante au monde pour moi. Mi señora madre, maman, hermosa. Merci beaucoup de m’avoir donné la vie, de m’avoir élevé, d’avoir fait de moi la personne forte, dédiée et surtout heureuse que je suis. C’est grâce à toi que mes mains sont remplies de tous les fruits que j’ai pu semer pendant ma vie. J’espère pouvoir te donner tout ce que tu mérites pour tout l’amour que tu m’as donné et pour tous les sacrifices que tu as faits. Tout ce que je suis, ce que j’ai, que je souhaite c’est pour toi, vis pour toi, respire pour toi. T’avoir aussi loin de moi implique que tu me manques à un point qui parfois m’est insupportable mais je sais que tu seras toujours ici avec moi, et que tu saches que je serai toujours là pour toi. Je t’aime avec tout mon être.

Lidi, merci énormément d’avoir été comme une deuxième mère pour moi, merci de toujours m’avoir fait sentir comme un de plus dans ta famille, tu m’as donné tellement d’amour que parfois je ne sais pas comment te le donner. Tu as été la première à me montrer ce monde de l’autre côté du “charco”. Tu as toujours été là pour me câliner, me gâter et m’encourager dans tellement de choses. Je t’aime idi.

Fa, papa, merci beaucoup pour tout ce que tu m’as donné, pour la patience et l’amour que tu as toujours eu pour moi. Quelque chose qui m’a toujours donné, et qui me donne encore aujourd’hui tellement de sécurité est de savoir que je peux toujours compter sur toi, que tu seras toujours là quand j’en ai besoin, et que, même si nos caractères choquent constamment (je l’ai hérité de toi hahaha), tu m’aimes toujours de toutes tes forces. Je t’aime.

Merci beaucoup Mar car grandir à tes côtés a été comme avoir une soeur qui me protégeait et qui prenait soin de moi. Tu ne sais pas à quel point tu m’as rendu heureux avec toute la gentillesse que tu éprouvais pour moi. Grâce à toi j’ai compris tellement de choses et ma vie ne serait pas la même aujourd’hui sans cela. Marco, German, merci de m’avoir traité comme un frère (avec tout le bon et mauvais) parce que grâce à ça je me sentais moins seul.

Mari, prima, Mariana, ton âme, ta présence, ton être me font sentir tellement accompagné dans cet océan gigantesque que nous appelons la réalité. Nous pensons, ressentons et vivons la vie d’une façon tellement similaire que je suis sûr que c’est ce qui nous a rapproché autant depuis que nous étions enfants, et c’est ce qui a créé un lien aussi fort. Merci pour chaque seconde que tu as su me donner pour m’écouter, me soutenir et me donner tes conseils. Je t’aime tellement mais tellement que j’aimerai avoir ta plume pour pouvoir le décrire avec des mots. Continue de suivre ton feu intérieur et n’oublie jamais qui tu es, ce que tu as accompli et tout le pouvoir que tu as, parce que tu as changé ma vie radicalement.

Nay, Ive, je remercie la vie de nous avoir rapproché autant quand elle l’a fait. Je suis heureux de vous avoir comme mes soeurs et de l’amour que vous éprouvez envers moi à chaque occasion que vous avez. Danser avec vous, chanter avec vous, fêter avec vous a toujours eu une chaleur singulière qui me remplit de joie. Je vous adore.

Palola, Pola, Dude, mon enfance n’auraient pas été une vraie enfance sans toi, et mes poumons ne seraient pas aussi coriaces sans toutes les heures que l’on a passé à rire. Je ne crois pas connaître quelqu’un qui m’ait infligé autant de maux de ventre avec toutes ses bêtises. Ma tête est remplie de souvenirs à se disputer, à se pardonner, à faire des bêtises qui énervaient nos mères et faisaient rire nos grand-parents. Tu me manques beaucoup et je t’aime encore plus.

Je remercie mes tantes car, comme beaucoup le savent, c’est grâce à toutes ces femmes fortes que je suis qui je suis aujourd’hui. Merci beaucoup Lale de t’occuper de moi autant, et merci aussi pour tous ces films, ces telenovelas et ces rires avec Pao et ma mère. Maruka, nos nuits avec les “cositas de la mujer” me manquent tellement. Mes souvenirs sont remplis de petits moments avec toi que peut-être je n’ai pas su exprimer à quel point ils remplissent mon âme de douceur, de chaleur et de calme. Merci. Merci Kenuki y Elsuki car vous avez toujours su me faire sentir comme un enfant, et j’adore ça. Ça me manque beaucoup tous ces noëls dans la cuisine entre rires, cris et, mes préférés, ragots. Merci Ceci de toujours avoir été là et d’avoir partagé ton amour avec moi (et ton amour pour le gâteau à la fraise).

Je veux remercier mes grands-pères et grandes-mères qui ne sont plus avec nous mais qui m’ont toujours donné tout leur amour et câlins, chacun et chacune à sa manière. J’espère que, où vous serez, vous vous sentez bien et fiers.

\vspace{5mm}
Je vais finir ces remerciements avec deux personnes qui sont pour moi mes plus grands trésors.

Cosette, Cosa, la reine, ou comme les autres te connaissent, Carolina. Tu es la preuve que la famille nous la construisons, que la famille nous la choisissons, que la famille est faite des personnes qui sont toujours là pour nous. Tu es ma famille, tu es la personne qui me soutient, qui me dispute, qui m’écoute, qui me regarde dans les yeux pour me dire que “tout va se stabiliser”, car tu sais que ce sera toujours le cas. Ça fait tellement des années que je te connais que maintenant j’ai l’impression que tu as toujours été avec moi, que tu as toujours partagé ton rire, tes ragots, tes conseils de beauté mais, avant tout, ton être. La vie t’a amené aussi proche de moi et je suis tellement content de cela. Je ne serais pas là aujourd’hui sans toi. Je pourrais écrire mille paragraphes pour te remercier pour chaque belle chose que tu m’as donnée, pour chaque acte d’affection et d’amour, et même si je sais que tu aimerais ça, il n’y a pas assez de papier à Paris pour imprimer un tel manuscrit. Donc je vais tout résumer en une phrase: you are my person.

Gwen. Merci Gwen. Merci d’exister. Merci d’avoir accepté ce tout premier date. Merci pour les rire, les pleurs, les pigeons, les séries, les nuits de films d’horreur et les matins de café à regarder Bob’s. Tu es la chose la plus grande que j’ai. Tu es une personne remplie de gentillesse, d’amour, d’humanité et tu ne cesses pas de me montrer ta bienveillance chaque jour depuis que je te connais. Depuis ce premier jour tu as su être un soutien inconditionnel, tu as été là pendant l’écriture de ce manuscrit, et tu as pu voir à quel point ça a été difficile, mais c’est grâce à toi que j’ai réussi. Merci de me cuisiner, de m’écouter, de me câliner. Merci de rester avec moi quand j’en ai besoin, et de me donner la liberté que j’aime tant. Il n’y a pas un seul endroit sur terre où je me sens le plus en sécurité qu'entre tes bras. Merci d’être qui tu es. This nest is beautiful because it’s ours.