
\chapter{Discussion}

The Hawkes process model is used accross a vast multitude of application fields, and so establishing inference procedures to model more complex phenomena is undeniably important. 
Our work provided novel statistical tools for the estimation of Hawkes processes within the frequentist parametric framework.
Our effort was centered around establishing theoretical results for two different submodels, with code implementations of our procedures and numerical illustrations of their efficiency. 

\vspace{5mm}

% Part 1

The first part of our work focused on the study of inhibiting interactions for Hawkes processes by establishing the maximum likelihood estimation procedure.
In Chapter~\ref{chapter:univariate_inhibition}, we implemented this method under the assumption of monotony of the kernel function, which allowed us to derive a closed-form expression of the log-likelihood. 
In Chapter~\ref{chapter:multivariate_inhibition}, we extended our previous work to the multivariate setting for exponentially-shaped interactions which revealed the challenges of establishing the identifiability of our model.
This issue arised primarily from our use of the positive part function in the non-linear Hawkes process model, as the non-bijectivity of the function may hinder the accurate identification of interaction parameters.
The choice of this function helps modelling a closer behaviour to the classical linear Hawkes model and simplifies the computation of the intensity function.
Prioritising positive non-linear functions is a way to circumvent this problem, as shown in the Bayesian context in \textcite{Sulem2024}, for a shifted version of the positive part function.
By adapting our procedure to an approximation of the positive part functions, for example through the softplus function, the resulting method could be a satisfying compromise between model fidelity and theoretical guarantees. 

Numerically, we illustrated the improvements in our estimations by incorporating a support inference step for the interaction matrix, as evidenced by our goodness-of-fit hypothesis testing.
Our methods relied on data-based paradigms which are highly dependant on sample size.
In practical applications, such as medical data analysis with limited observations, it is important to propose more robust methods. 
Future research could explore Lasso penalisation procedures, in the same vein as the work of \textcite{Bacry2020}, where both an $\ell_1$ and trace-norm penalisations are tuned by the means of data-driven weights and that can benefit from cross-validation procedures.

In our application to the study of neuronal activity, the results exhibited a clear self-interaction behaviour for each studied neuron.
This is consistent with the known behaviour of neurons as they present a post-activation refractory period, reflecting an incapacity to reactivate 
after a synaptic impulse is produced.
The fact that our estimations are not consistently self-inhibitory is most likely due to the choice of a monotone kernel in the form of the exponential function.
In reality, when membrane potential is measured, the refractory effect is delayed which is better represented by a non-monotone function, suggesting that extending our work for more complex interaction functions is a clear path to obtaining more faithful models.

\vspace{5mm}

The second part of our work focused examined noisy observations of Hawkes processes with excitation. 
In Chapter~\ref{chapter:spectral_superposition}, we modelled the presence of exogeneous points by superposing a Hawkes process with a homogeneous Poisson process.
Chapter~\ref{chapter:spectral_thinning} adresses the absence of points through a thinning representation of the original point process.
Our spectral approach, utilising the Bartlett spectrum and periodogram, enabled us to establish an estimation procedure by optimising a spectral version of the log-likelihood.
The main difficulty encountered was the identifiability issues that we showed derive directly from the spectral density function definition.
To partially solve this problem, we provided some conditions on the kernel functions that allowed us to recover the identifiability of our model.
Our method will greatly benefit from more general results, particularly for higher dimensional processes with more complex interactions.
Another possible solution could be found in the recent works of \textcite{Roueff2019} that establish an equivalent expression of the Bartlett spectrum by relaxing the stationarity condition to a local stationarity. 
This could help to incorporate more general shapes of noise which in turn could help to better identify the noise dynamics.

Our numerical illustrations in synthetic data showed the efficiency of our approach to perform this difficult estimation task and the next step is to apply our results to real-world data.
This can help to provide a better insight in contexts where measurement errors are more likely to appear.
In particular, it can be interesting to combine both models studied in Chapters~\ref{chapter:spectral_superposition} and \ref{chapter:spectral_thinning}% and results 
in order to study the spread of illnesses through screening.
In this context, the presence of false positives and false negatives can be modelled by the superposition and thinning operations and so our model can clearly help to obtain better interpretations of disease transmission dynamics.

Finally, in the context of subsampling through the thinning operation, a significant issue concerns the lack of evaluation methods for spectral estimators in unsupervised settings.
As suggested in the recent works of \textcite{Cronie2024, Coeurjolly2024}, subsampling paradigms can greatly improve the estimation procedures and allow for more complex methods as cross-validation paradigms to be available for the study of point processes. 
Establishing asymptotic results for our estimators could partially address this issue by providing a way of establishing asymptotic confidence intervals. Encouragingly, there has been recent advances concerning these guarantees for the periodogram in the works of \textcite{Rajala2023, Yang2024} which pave the way for the study of the periodogram.

