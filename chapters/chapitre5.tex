\leadchapter{
  ABSTRACT
}

\chapter[][]{A numerical exploration of spectral methods for thinned Hawkes processes}

\section{Introduction}

INTRODUCTION

\section{Mathematical setting}\label{sec:chap5_mathsetting}

\subsection{The $p$-thinning of point processes}\label{sec:chap5_hawkesprocess}

In this chapter, we will focus our study on the stationary univariate Hawkes process on the real line $\RR$, noted $H$.
The Hawkes process can be characterised by its conditional intensity function, for all $t\in\RR$:
\[\lambda(t) = \mu + \int_{-\infty}^{t}{h(t-s)\,H(\dd s)} = \mu + \sum_{T_k \leq t}{h(t-T_k)}\,,\]
where $\mu > 0$ is the baseline intensity, and $h\colon\RR\to \RR_{\geq 0}$ is the kernel function modelling the self-exciting behaviour of past points $(T_k)_{k\in\ZZ}$.

In order for $H$ to be a point process (\ie an a.s. finite measure in any bounded set $B$), 
a sufficient and necessary condition \parencite{Hawkes1971} is that 
\[\|h\|_1 = \int_{\RR}{h(t)\,\dd t} < 1\,.\]
Furthermore, let $\mathcal{B}_{\RR}^c$ denote the set of bounded Borel sets on $\RR$, and let, for any $B\in\mathcal{B}_{\RR}^c$:
\[H(B) = \sum_{k\in\ZZ}{\II_{T_k \in B}}\,,\]
be the number of event times in $B$. 

In this chapter, we will study a thinned version of process $H$ defined as follows:
\begin{definition}
For any $p\in(0,1)$, we denote $H_p$ a $p$-thinning of a point process $H$, defined for any $B\in\mathcal{B}_{\RR}^c$ as:
\[H_p(B) = \sum_{k\in\ZZ}{\II_{T_k \in B} B_k}\,,\]
where $(Z_k)$ is an i.i.d. collection of Bernoulli random variables of parameter $p$.
\end{definition}

In practice, the event times of process $H_p$ correspond to a subset of $(T_k)_{k\in\ZZ}$ where each point
is erased with a random probability $1-p$. 

\subsection{Spectral theory for point processes}

For a point process $H$, we define the first and second-order measures $M_1, M_2$, for any $A, B\in\mathcal{B}_{\RR}^c$, as:
\[M_1(A) = \EE[H(A)]\,,\qquad M_2(A,B) = \EE[H(A)H(B)]\,.\]
Under stationarity conditions, it follows that \parencite[Proposition 8.1.I]{DaleyV1}, for any $A, B\in\mathcal{B}_{\RR}^c$:
\[M_1(A) = m_1 \ell_{\RR}(A)\,,\]
where $m_1 = \EE[H([0,1])]$ is usually known as the average intensity of process $H$.

The spectral analysis approach of point processes is based on the Bartlett spectrum $\Gamma$ \parencite{Bartlett1963}, 
a measure on $\RR$ associated to the second-order moment measure $M_2$. 
To introduce it properly, let us consider the Schwartz space $\mathcal{S}$ defined as:
\[
    \mathcal S = \left\{ f \in C^\infty, \forall k \in \{1, 2, \dots\}, \forall r \in \{1, 2, \dots\},
    \sup_{x \in \RR} \left| x^r
    f^{(k)}(x)
    \right| < \infty \right\}\,,
\] where $C^\infty$ denotes the set of infinitely differentiable functions $f$ from $\RR$ to $\RR$, 
and $f^{(k)}$ denotes the $k^{th}$-order derivative of $f$.
In particular, for any $f\in\mathcal{S}$, we define its Fourier transform $\tilde f$, for any $\omega\in\RR$, as:
\[\tilde f (\omega) = \int_{\RR}{f(x) \mathrm{e}^{-2\pi\iu x \omega}\,\dd x}\,.\]

The Bartlett spectrum of a point process $H$ is the measure $\Gamma$ such that, 
for any $\varphi\in\mathcal{S}$ \parencite[Definition 2]{Bremaud2005}:
\begin{equation}\label{eq:chap5_bartlett_variance}
    \VV\left[\int_{\RR}{\varphi(x)H(\dd x)}\right] = \int_{\RR}{|\tilde\varphi(\omega)|^2\Gamma(\dd \omega)}\,.
\end{equation}
Existence of such a measure is esablished for any stationary point process \parencite[Proposition 8.2.I.(a)]{DaleyV1},
and by polarisation of Equation~\eqref{eq:chap5_bartlett_variance}, we obtain for any $\varphi, \psi \in \mathcal{S}$:
\begin{equation}\label{eq:chap5_bartlett_covariance}
v\end{equation}

Whenever $\Gamma$ is an absolutely continous measure, we denote $f:\RR\to \CC$ its Radon-Nikodym derivative, 
known as the spectral density of process $H$.

Another important quantity in the spectral theory of point processes is the periodogram $I^T\colon \RR \to \CC$:
for a realisation $(T_k)_{k=1:N(T)}$ of a process $H$ in a time window $[0,T]$, the periodogram is defined, for all $\omega\in\RR$, as:
\[I^T(\omega) = \frac{1}{T}\sum_{k=1}^{N(T)}\sum_{l=1}^{N(t)}{\mathrm{e}^{-2\pi \iu \omega (T_k - T_l)}}\,.\]

Furthermore, for any sequence $(\omega_k)_{k=1:M}$, such that $\omega_k \neq \omega_l$, for all integers $k\neq l$,
the random variables $(I^T(\omega_k))_{k=1:M}$ are asymptotically independent and exponentially distributed \parencite{Tuan1981} with respective parameter $(1/f(\omega_k))_{k=1:M}$.

In this chapter, we will work on a parametric setting, 
so let us assume that we dispose of a statistical model for the spectral density defined as:
\[
    \mathcal{P} = \left\{
        f_\theta \colon \RR \to \CC \colon \theta \in \Theta
    \right\}\,.
\]
We can then define the \textit{spectral} log-likelihood $\ell_T(\theta)$ for an observation $(T_k)_{k=1:N(T)}$ of $H$ as:
\begin{equation}\label{eq:chap5_spectral_ll}
    \ell_T(\theta) = -\frac{1}{T}\sum_{k=1}^{M}{\log(f_\theta (\omega_k)) + \frac{I^T(\omega_k)}{f_\theta(\omega_k)}}\,.
\end{equation}

We can then introduce the Whittle estimator $\hat \theta$ \parencite{Whittle1952} as:
\[\hat \theta = \arg\max_{\theta \in \Theta}~ \ell_T (\theta)\,.\]

\section{Parametric estimation of a thinned process}\label{sec:chap5_estimation}

\subsection{The spectrum of a thinned point process}

The goal of this chapter is to establish a parametric estimation method for the observation of a thinned Hawkes process.
We will then begin by establishing the expression of a $p$-thinning of a point process. 
We will leverage the study of spectral quantities on marked point process, 
which can be found in \textcite{Bremaud2002, Bremaud2005}. 

For this, let us introduce an alternative way of viewing the $p$-thinning of a point process through marked point process theory.
In this section, $H$ will denote a stationary point process on $\RR$ with spectral density function $f$.

We define the marked point process $\bar H$ associated to $H$, with marks $Z_k$ on a metric space $\mathcal{K}$ as the collection of points
$(T_k, Z_k)_{k\in\ZZ} \in (\RR \times \mathcal{K})^{\ZZ}$ (see \textcite[Chapter 6.4]{DaleyV1} for a more thorough presentation of marked point processes). 
The random marks $Z_k$ are usually used to represent underlying information on the event times of a point process $H$, 
which is often referred to as the \textit{ground} process.
In our work, we will restrict ourselves to the case where the random variables $(Z_k)_{k\in\ZZ}$ are independent and identically distributed.
In this way, process $\bar H$ is well-defined \parencite[6.4.IV(a)]{DaleyV1}.

We can then view a $p$-thinning of $H$ as a marked version $\bar H$ where $\mathcal{K} = [0,1]$
and the $(Z_k)_{k\in\ZZ}$ are a collection of Bernoulli random variables of parameter $p$.
This way we may define the thinned process $H_p$, for any $B\in\mathcal{B}^c$, as:
\[H_p(B) = \bar H(B\times\{1\})\,.\]

Under this scope, we will apply the results of \textcite{Bremaud2005}, 
that we adapt to our notations here as:

\begin{theorem}[{\textcite[Theorem 2]{Bremaud2005}}]
    Let $H$ be a stationary point process with Bartlett spectrum measure $\Gamma$ and 
    let $\bar H$ be a marked version of $H$ with i.i.d. marks $Z_k$ with shared distribution $Z$ on a metric space $\mathcal{K}$.
    Let $\varphi^\star, \psi^\star$ be measurable functions from $\RR \times \mathcal{K} \to \RR$, 
    such that:
    \begin{itemize}
        \item $\displaystyle
            \int_{\RR}{\EE\left[|\varphi^\star(x, Z)|\dd(x)\right]} < +\infty\,,\qquad \int_{\RR}{\EE\left[|\psi^\star(x, Z)|\dd(x)\right]} < +\infty\,.
        $
        \item $\displaystyle
            \int_{\RR}{\EE\left[\varphi^\star(x, Z)^2\dd(x)\right]} < +\infty\,,\qquad\int_{\RR}{\EE\left[\psi^\star(x, Z)^2\dd(x)\right]} < +\infty\,.
        $
        \item By denoting $\bar \varphi : x\to \EE\left[\varphi^\star(x, Z)\right]$ and $\bar \psi : x\to \EE\left[\psi^\star(x, Z)\right]$, then,
              \[\bar \varphi,\bar \psi \in\mathcal{S}\,.\]
    \end{itemize}

    Then, it follows that:
    \begin{equation}\label{eq:chap5_marked_spectrum}
        \Cov\left(\sum_{k\in\ZZ}{\varphi^\star(T_k, Z_k)}, \sum_{k\in\ZZ}{\psi^\star(T_k, Z_k)}\right) = \int_{\RR}{\tilde {\bar\varphi} (\omega) \tilde{\bar \psi} (-\omega)\,\Gamma(\dd \omega)} 
        + \int_{\RR}{\Cov\left( \tilde \varphi(\omega, Z), \tilde \psi (\omega, Z)\,M_1(\dd \omega)\right)}\,.
    \end{equation}
\end{theorem}

The utility of this result resides on the link that it established between the covariance of a marked point process $\bar H$
and the Bartlett spectrum of its ground process $H$.

\subsection{The $p$-thinned Hawkes process estimation}