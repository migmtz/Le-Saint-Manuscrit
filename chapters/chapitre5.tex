\leadchapter{
  ABSTRACT
}

\chapter[][]{A numerical exploration of spectral methods for thinned Hawkes processes}

\section{Introduction}

INTRODUCTION
MOTIVATION = EN UTILISANT $p$ ON PEUT FAIRE UNE SORTE DE SUBSAMPLING OU CROSS-VALIDATION

\section{Mathematical setting}\label{sec:chap5_mathsetting}

\subsection{The $p$-thinning of point processes}\label{sec:chap5_hawkesprocess}

In this chapter, we will focus our study on the stationary univariate Hawkes process on the real line $\RR$, noted $H$.
The Hawkes process can be characterised by its conditional intensity function, for all $t\in\RR$:
\begin{equation}\label{eq:chap5_hawkes_intensity}
    \lambda(t) = \mu + \int_{-\infty}^{t}{h(t-s)\,H(\dd s)} = \mu + \sum_{T_k \leq t}{h(t-T_k)}\,,
\end{equation}
where $\mu > 0$ is the baseline intensity, and $h\colon\RR\to \RR_{\geq 0}$ is the kernel function modelling the self-exciting behaviour of past points $(T_k)_{k\in\ZZ}$.

In order for $H$ to be a point process (\ie an a.s. finite measure in any bounded set $B$), 
a sufficient and necessary condition \parencite{Hawkes1971} is that 
\[\|h\|_1 = \int_{\RR}{h(t)\,\dd t} < 1\,.\]
Furthermore, let $\mathcal{B}_{\RR}^c$ denote the set of bounded Borel sets on $\RR$, and let, for any $B\in\mathcal{B}_{\RR}^c$:
\[H(B) = \sum_{k\in\ZZ}{\II_{T_k \in B}}\,,\]
be the number of event times in $B$. 

In this chapter, we will study a thinned version of process $H$ defined as follows:
\begin{definition}\label{def:chap5_thinning}
For any $p\in(0,1)$, we denote $H_p$ a $p$-thinning of a point process $H$, defined for any $B\in\mathcal{B}_{\RR}^c$ as:
\[H_p(B) = \sum_{k\in\ZZ}{\II_{T_k \in B} Z_k}\,,\]
where $(Z_k)$ is an i.i.d. collection of Bernoulli random variables of parameter $p$.
\end{definition}

In practice, the event times of process $H_p$ correspond to a subset of $(T_k)_{k\in\ZZ}$ where each point
is erased with a random probability $1-p$. 

\subsection{Spectral theory for point processes}\label{sec:chap5_spectral_theory}

For a point process $H$, we define the first and second-order measures $M_1, M_2$, for any $A, B\in\mathcal{B}_{\RR}^c$, as:
\[M_1(A) = \EE[H(A)]\,,\qquad M_2(A,B) = \EE[H(A)H(B)]\,.\]
Under stationarity conditions, it follows that \parencite[Proposition 8.1.I]{DaleyV1}, for any $A, B\in\mathcal{B}_{\RR}^c$:
\[M_1(A) = m_1 \ell_{\RR}(A)\,,\]
where $m_1 = \EE[H([0,1])]$ is usually known as the average intensity of process $H$.

The spectral analysis approach of point processes is based on the Bartlett spectrum $\Gamma$ \parencite{Bartlett1963}, 
a measure on $\RR$ associated to the second-order moment measure $M_2$. 
To introduce it properly, let us consider the Schwartz space $\mathcal{S}$ defined as:
\[
    \mathcal S = \left\{ f \in C^\infty, \forall k \in \{1, 2, \dots\}, \forall r \in \{1, 2, \dots\},
    \sup_{x \in \RR} \left| x^r
    f^{(k)}(x)
    \right| < \infty \right\}\,,
\] where $C^\infty$ denotes the set of infinitely differentiable functions $f$ from $\RR$ to $\RR$, 
and $f^{(k)}$ denotes the $k^{th}$-order derivative of $f$.
In particular, for any $f\in\mathcal{S}$, we define its Fourier transform $\tilde f$, for any $\omega\in\RR$, as:
\[\tilde f (\omega) = \int_{\RR}{f(x) \mathrm{e}^{-2\pi\iu x \omega}\,\dd x}\,.\]

The Bartlett spectrum of a point process $H$ is the measure $\Gamma$ such that, 
for any $\varphi\in\mathcal{S}$ \parencite[Definition 2]{Bremaud2005}:
\begin{equation}\label{eq:chap5_bartlett_variance}
    \VV\left[\int_{\RR}{\varphi(x)H(\dd x)}\right] = \int_{\RR}{|\tilde\varphi(\omega)|^2\Gamma(\dd \omega)}\,.
\end{equation}
Existence of such a measure is esablished for any stationary point process \parencite[Proposition 8.2.I.(a)]{DaleyV1},
and by polarisation of Equation~\eqref{eq:chap5_bartlett_variance}, we obtain for any $\varphi, \psi \in \mathcal{S}$:
\begin{equation}\label{eq:chap5_bartlett_covariance}
    \Cov\left(\int_{\RR}{\varphi(x)H(\dd x)}, \int_{\RR}{\psi(x)H(\dd x)}\right) = \int_{\RR}{\tilde \varphi(\omega)\tilde \psi(-\omega)\Gamma(\dd \omega)}\,.
\end{equation}

Whenever $\Gamma$ is an absolutely continous measure, we denote $f:\RR\to \CC$ its Radon-Nikodym derivative, 
known as the spectral density of process $H$.

Another important quantity in the spectral theory of point processes is the periodogram $I^T\colon \RR \to \CC$:
for a realisation $(T_k)_{k=1:N(T)}$ of a process $H$ in a time window $[0,T]$, the periodogram is defined, for all $\omega\in\RR$, as:
\[I^T(\omega) = \frac{1}{T}\sum_{k=1}^{N(T)}\sum_{l=1}^{N(t)}{\mathrm{e}^{-2\pi \iu \omega (T_k - T_l)}}\,.\]

Furthermore, for any sequence $(\omega_k)_{k=1:M}$, such that $\omega_k \neq \omega_l$, for all integers $k\neq l$,
the random variables $(I^T(\omega_k))_{k=1:M}$ are asymptotically independent and exponentially distributed \parencite{Tuan1981} with respective parameter $(1/f(\omega_k))_{k=1:M}$.

In this chapter, we will work on a parametric setting, 
so let us assume that we dispose of a statistical model for the spectral density defined as:
\[
    \mathcal{P} = \left\{
        f_\theta \colon \RR \to \CC \colon \theta \in \Theta
    \right\}\,.
\]
We can then define the \textit{spectral} log-likelihood $\ell_T(\theta)$ for an observation $(T_k)_{k=1:N(T)}$ of $H$ as:
\begin{equation}\label{eq:chap5_spectral_ll}
    \ell_T(\theta) = -\frac{1}{T}\sum_{k=1}^{M}{\log(f_\theta (\omega_k)) + \frac{I^T(\omega_k)}{f_\theta(\omega_k)}}\,.
\end{equation}

We can then introduce the Whittle estimator $\hat \theta$ \parencite{Whittle1952} as:
\[\hat \theta = \arg\max_{\theta \in \Theta}~ \ell_T (\theta)\,.\]

\section{Parametric estimation of a thinned process}\label{sec:chap5_estimation}

\subsection{The spectrum of a thinned point process}

The goal of this chapter is to establish a parametric estimation method for the observation of a thinned Hawkes process.
We will then begin by establishing the expression of a $p$-thinning of a point process. 
We will leverage the study of spectral quantities on marked point process, 
which can be found in \textcite{Bremaud2002, Bremaud2005}. 

For this, let us introduce an alternative way of viewing the $p$-thinning of a point process through marked point process theory.
In this section, $H$ will denote a stationary point process on $\RR$ with spectral density function $f$.

We define the marked point process $\bar H$ associated to $H$, with marks $Z_k$ on a metric space $\mathcal{K}$ as the collection of points
$(T_k, Z_k)_{k\in\ZZ} \in (\RR \times \mathcal{K})^{\ZZ}$ (see \textcite[Chapter 6.4]{DaleyV1} for a more thorough presentation of marked point processes). 
The random marks $Z_k$ are usually used to represent underlying information on the event times of a point process $H$, 
which is often referred to as the \textit{ground} process.
In our work, we will restrict ourselves to the case where the random variables $(Z_k)_{k\in\ZZ}$ are independent and identically distributed.
In this way, process $\bar H$ is well-defined \parencite[6.4.IV(a)]{DaleyV1}.

We can then view a $p$-thinning of $H$ as a marked version $\bar H$ where $\mathcal{K} = [0,1]$
and the $(Z_k)_{k\in\ZZ}$ are a collection of Bernoulli random variables of parameter $p$.
This way we may define the thinned process $H_p$, for any $B\in\mathcal{B}^c$, as:
\[H_p(B) = \bar H(B\times\{1\})\,.\]

Under this scope, we will apply the results of \textcite{Bremaud2005}, 
that we adapt to our notations here as:

\begin{theorem}[{\textcite[Theorem 2]{Bremaud2005}}]\label{th:chap5_spectral_marked}
    Let $H$ be a stationary point process with Bartlett spectrum measure $\Gamma$ and 
    let $\bar H$ be a marked version of $H$ with i.i.d. marks $Z_k$ with shared distribution $Z$ on a metric space $\mathcal{K}$.
    Let $\varphi^\star, \psi^\star$ be measurable functions from $\RR \times \mathcal{K} \to \RR$, 
    such that:
    \begin{itemize}
        \item $\displaystyle
            \int_{\RR}{\EE\left[|\varphi^\star(x, Z)|\right]\dd(x)} < +\infty\,,\qquad \int_{\RR}{\EE\left[|\psi^\star(x, Z)|\right]\dd(x)} < +\infty\,.
        $
        \item $\displaystyle
            \int_{\RR}{\EE\left[\varphi^\star(x, Z)^2\right]\dd(x)} < +\infty\,,\qquad\int_{\RR}{\EE\left[\psi^\star(x, Z)^2\right]\dd(x)} < +\infty\,.
        $
        \item By denoting $\bar \varphi : x\to \EE\left[\varphi^\star(x, Z)\right]$ and $\bar \psi : x\to \EE\left[\psi^\star(x, Z)\right]$, then,
              \[\bar \varphi,\bar \psi \in\mathcal{S}\,.\]
    \end{itemize}

    Then, it follows that:
    \begin{equation}\label{eq:chap5_marked_spectrum}
        \Cov\left(\sum_{k\in\ZZ}{\varphi^\star(T_k, Z_k)}, \sum_{k\in\ZZ}{\psi^\star(T_k, Z_k)}\right) = 
        \int_{\RR}{\tilde {\bar\varphi} (\omega) \tilde{\bar \psi} (-\omega)\,\Gamma(\dd \omega)} 
        + \int_{\RR}{\Cov\left( \tilde \varphi^\star(\omega, Z), \tilde \psi^\star (-\omega, Z)\right)M_1(\dd \omega)}\,,
    \end{equation}
    where, for any $\omega\in\RR$, $\tilde \varphi^\star(\omega, Z)$ (resp. $\tilde \psi^\star(\omega, Z)$) denotes the Fourier transform of the function $x\to \varphi^\star(x, Z)$ (resp. $x\to \psi^\star(x, Z)$).
\end{theorem}

The utility of this result resides on the link that it established between the covariance of a marked point process $\bar H$
and the Bartlett spectrum of its ground process $H$. 
This allows to establish an explicit expression of the spectral density of the $p$-thinning of a process $H$, 
as presented in the following proposition:

\begin{proposition}\label{prop:chap5_spectral_thinning}
    Let $H$ be a stationary point process admitting a Bartlett spectrum $\Gamma^N$ defined as in Equation~\eqref{eq:chap5_bartlett_variance}.
    We assume that $\Gamma$ is absolutely continuous and we note $f$ the spectral density function. 
    Let $m_1$ be the average intensity of $H$.

    For any $p\in(0,1)$, let $H_p$ be a $p$-thinning of $H$ as explicited in Definition~\ref{def:chap5_thinning}.

    Then, $H_p$ admits a spectral density function, denoted $f_p$, such that for any $\omega\in\RR$:
    \begin{equation}\label{eq:chap5_spectral_thinning}
        f_p(\omega) = p^2 f(\omega) + p(1-p)m_1\,.
    \end{equation}

\end{proposition}

\begin{proof}
    The proof is given in Appendix~\ref{appendix:chap5_proof_spectral_thinning}
\end{proof}

    Let us remark that Equation~\eqref{eq:chap5_spectral_thinning} can be found in \textcite[Equation 8.3.5]{DaleyV1},
    their spectral density function is $\gamma$, $m$ corresponds to the average intensity of $H$ and $\pi_2 = p$.
    The thinning scenario presented in this chapter is obtained for the following values:
    \[\mu_2 = 0\,, \qquad G\equiv 1\,,\]
    the missing factor $1/(2\pi)$ corresponding to a different expression for the Fourier transform chosen by the authors.

    The authors obtain their result by term identification of a bivariate point process. 
    The proof presented in this chapter is an alternative way of establishing this result,
    illustrating the usefulness of leveraging the spectral theory of point processes.

\subsection{The $p$-thinned Hawkes process estimation and the exponential kernel}

    For the rest of this chapter, 
    $H$ will denote a stationary Hawkes process defined by the intensity $\lambda$ as in Equation~\eqref{eq:chap5_hawkes_intensity}.

    As shown in \textcite{Hawkes1971}, the spectral density of a Hawkes process reads:
    \[f(\omega) = \frac{\mu}{(1-\|h\|_1)|1-\tilde h(\omega)|^2}\,.\]

    Corollary~\ref{cor:chap5_hawkes_thinning}, presents the spectral density of the $p$-thinned univariate Hawkes process:

    \begin{corollary}\label{cor:chap5_hawkes_thinning}
        Let $H$ be a stationary univariate Hawkes process with baseline intensity $\mu > 0$, and kernel function $h\colon \RR\to \RR_{>0}$,
        as defined by Equation~\eqref{eq:chap5_hawkes_intensity}. Let $p\in(0,1)$ and $H_p$ a $p$-thinning of $H$. 
        Then the spectral density $f_p$ of $H_p$ is:
        \begin{equation}\label{eq:chap5_spectral_hawkes}
            \forall \omega \in \RR\,, \quad f_p(\omega) = p^2 \frac{\mu}{(1-\|h\|_1)|1-\tilde h(\omega)|^2} + p(1-p)\frac{\mu}{1-\|h\|_1}\,.
        \end{equation}
    \end{corollary}
    \begin{proof}
        The proof is direct by applying Proposition~\ref{prop:chap5_spectral_thinning} and 
        with the expression of the Hawkes process spectral density.
    \end{proof}
    We can define the statistical model:
    \[\mathcal{P} = 
      \left\{
        f_\theta \colon \RR \to \CC, \theta = (\mu, \gamma, p) \in \Theta
      \right\}\,.
    \]

    As mentioned previously, it is then possible to estimate $\theta$ by maximising the spectral log-likelihood (Equation~\eqref{eq:chap5_spectral_ll}).
    Let us now focus on the exponential interaction function, defined as:
    \[\forall t\in\RR\,,\quad h(t) = \alpha \beta \mathrm{e}^{-\beta t}\II_{t\geq 0}\,,\]
    where $\alpha \in(0,1)$ and $\beta > 0$.

    As the Fourier transform is explicit and reads:
    \[\tilde h (\omega) = \frac{\alpha \beta}{\beta + 2 \pi \iu \omega}\,,\]
    and so Equation~\eqref{eq:chap5_spectral_hawkes} reduces to:
    \begin{equation}\label{eq:chap5_spectral_exponential}
        \forall \omega\in\RR\,,\quad
        f_p(\omega) = \frac{\mu p }{1-\alpha}\left(1 + p \frac{\beta^2 \alpha (2-\alpha)}{\beta^2(1-\alpha)^2 + 4 \pi^2 \omega^2}\right)\,.
    \end{equation}

    In this context, the statistical model of $f_p$ is:
    \[\mathcal{Q} = 
      \left\{
        f_\theta \colon \RR \to \CC, 
        \theta = (\mu, \alpha, \beta, p) \in \RR_{>0}\times (0,1) \times \RR_{>0}\times(0,1)
      \right\}\,.
    \]
    Similar to the superposition case shown in \textcite[Proposition 3.2]{Bonnet2024}, 
    model $\mathcal{Q}$ is not identifiable in the general setting but the identifiability can be established, as shown in Proposition~\ref{prop:chap5_identifiability_thinning}, as long as one parameter of the model is fixed.

    \begin{proposition}\label{prop:chap5_identifiability_thinning}
        The model $\mathcal{Q}$ is identifiable if and only if one of the parameters in the 4-uplet $(\mu, \alpha, \beta, p)$ is fixed.

        In particular, for any admissible parameter $\theta = (\mu, \alpha, \beta, p)$,
        for any $\kappa\in(0, 1/p)$, let:
        \begin{equation}\label{eq:chap5_nonidentifiable_parameters}
        \begin{cases}
            \mu' = \frac{\mu}{\kappa(1-\alpha)\sqrt{1 + \frac{1}{\kappa}\left(\frac{1}{(1-\alpha)^2} - 1\right)}}\\
            \alpha' = 1 - \frac{1}{\sqrt{1 + \frac{1}{\kappa}\left(\frac{1}{(1-\alpha)^2} - 1\right)}}\\
            \beta' = \beta (1-\alpha) \sqrt{1 + \frac{1}{\kappa}\left(\frac{1}{(1-\alpha)^2} - 1\right)}\\
            p' = \kappa p\,.
          \end{cases}
        \end{equation}
          Then, $\theta' = (\mu', \alpha', \beta', p')$ is an admissible parameter such that $f_{\theta} = f_{\theta'}$.
    \end{proposition}

    \begin{proof}
        The proof is given in Appendix~\ref{appendix:chap5_proof_identifiability_thinning}
    \end{proof}

    The non-identifiability of the full model (four unknown parameters) limits the implementation of an estimation method for $\theta\in\Theta$, but this problem is avoided as long as one of the parameters is known beforehand. 
    In this work we will focus on the scenario where the value of $p$ is known in advance, 
    and so the previous result ensures that, for any $p^\star\in(0,1)$, the reduced model:
    
    \[\mathcal{Q}_{p} = 
        \left\{
            f_\theta \colon \RR \to \CC, 
            \theta = (\mu, \alpha, \beta, p) \in \RR_{>0}\times (0,1) \times \RR_{>0}\times(0,1), p=p^\star
        \right\}\,,
    \]
    is identifiable.
    We can then define an estimator of $\hat \theta = (\hat \mu, \hat \alpha, \hat \theta)$, as presented at the end of Section~\ref{sec:chap5_spectral_theory}, as:

    \[
        \hat \theta = \arg\max_{\theta \in \Theta} \ell_T (\theta)
        = \arg\max_{\theta \in \Theta}  \left(-\frac{1}{T}\sum_{k=1}^{M}{\log(f_\theta (\omega_k)) + \frac{I^T(\omega_k)}{f_\theta(\omega_k)}}\right)\,.
    \]

\section{Numerical illustrations}
    In this section, we will focus in 


\begin{subappendices}
    \section{Proof of Proposition~\ref{prop:chap5_spectral_thinning}}\label{appendix:chap5_proof_spectral_thinning}
        Let $H_p$ be a $p$-thinning of a stationary point process $H$ admitting a spectral density function $f$.
        The stationarity of $H_p$ is given by the fact that the variables $(Z_k)_{k\in\ZZ}$ are i.i.d. and so for any integer $r$, 
        for any $B_1\ldots, B_r \in\mathcal{B}^c$, 
        the random vector $(N_p(B_k)_{k=1:r})$ has the same distribution as $(N_p(B_k+t)_{k=1:r})$,
        for any $t\in\RR$.
        We will then denote $\Gamma_p$ and $f_p$ respectively the Bartlett spectrum and spectral density function of $H_p$

        In order to establish Equation~\eqref{eq:chap5_spectral_thinning}, we will leverage Theorem~\ref{th:chap5_spectral_marked}.
        We consider then the marked version $\bar H$ of $H$ where the marks $Z_k$ are i.i.d. Bernoulli distributions with common probability $p$.

        Let $\varphi, \psi \in \mathcal{S}$, we define, for all $x,z\in\RR\times\{0,1\}$, the functions:
        \[\varphi^\star(x,z) = \varphi(x)z\,,\qquad \psi^\star(x,z) = \psi(x)z\,.\]

        Let us verify that these functions verify the conditions of Theorem~\ref{th:chap5_spectral_marked}. 
        Without lose of generality, we will work uniquely with $\varphi^\star$, as the arguments are exactly the same for $\psi^\star$.
        
        For any $x\in\RR$, $\varphi^\star(x, Z) = \varphi(x)Z$ for Z a Bernoulli distribution of parameter $p$.
        It follows that $\varphi^\star(x, Z)$ admits a first and second order moment, and as $\varphi\in\mathcal{S}$, 
        $\varphi(x)Z$ is integrable and twice integrable, which shows that:
                \[
                \int_{\RR}{\EE\left[|\varphi^\star(x, Z)|\right]\dd(x)} < +\infty\,,\qquad 
                \int_{\RR}{\EE\left[\varphi^\star(x, Z)^2\right]\dd(x)} < +\infty\,.
                \]
        Furthermore, for any $x\in\RR$, $\bar \varphi(x) = \varphi(x)\EE[Z] = p \varphi(x)$,
        and so, as the Schwartz space is closed under scalar mutliplication, it follows that $\bar \varphi\in\mathcal{S}$.

        We can then apply Equation~\eqref{eq:chap5_marked_spectrum} to our marked process $\bar H$. 
        For this, let us notice that:
        \[
            \sum_{k\in\ZZ}{\varphi^\star(T_k, Z_k)} = \sum_{k\in\ZZ}{\varphi(T_k)Z_k} = \int_{\RR}{\varphi(t) H_p(\dd t)}\,,
        \]
        with the same expression holding for $\psi^\star$ and $\psi$. 
        So, the left-hand side of Equation~\eqref{eq:chap5_marked_spectrum} reads:
        \begin{align}
            \Cov\left(\sum_{k\in\ZZ}{\varphi^\star(T_k, Z_k)}, \sum_{k\in\ZZ}{\psi^\star(T_k, Z_k)}\right) &=
            \Cov\left(\int_{\RR}{\varphi(t) H_p(\dd t)}, \int_{\RR}{\psi(t) H_p(\dd t)}\right) \nonumber\\
            &= \int_{\RR}{\tilde \varphi(\omega)\tilde \psi(-\omega)\,\Gamma_p(\dd\omega)}\,,\label{eq:chap5_leftside_proof_thinning}
        \end{align}
        where the last equality comes from Equation~\eqref{eq:chap5_bartlett_covariance}.

        For the left hand side, let us remark that $\bar \varphi(x) = p \varphi(x)$ for any $x\in\RR$ and so, for any $\omega\in\RR$,
        \[
            \tilde \bar \varphi(\omega) = p \tilde \varphi(\omega)\,,
        \]
        and,
        \[\tilde \phi^\star(\omega, Z) = \tilde \phi(\omega)Z\,.\]

        The right-hand side of Equation~\eqref{eq:chap5_marked_spectrum} becomes:

        \begin{align}
            \int_{\RR}{\tilde {\bar\varphi} (\omega) \tilde{\bar \psi} (-\omega)\,\Gamma(\dd \omega)} 
        + &\int_{\RR}{\Cov\left( \tilde \varphi^\star(\omega, Z), \tilde \psi^\star (-\omega, Z)\,M_1(\dd \omega)\right)} \nonumber\\
        &= \int_{\RR}{p^2 \tilde {\varphi} (\omega) \tilde{\psi} (-\omega)\,\Gamma(\dd \omega)} 
        + \int_{\RR}{\tilde \varphi(\omega)\tilde \psi(\omega) \Cov\left(Z,Z\right)M_1(\dd \omega)} \nonumber\\
        &= \int_{\RR}{\tilde {\varphi} (\omega) \tilde{\psi} (-\omega)\,(p^2\Gamma(\dd \omega) + p(1-p) M_1(\dd \omega))}\,.\label{eq:chap5_rightside_proof_thinning}
        \end{align}

        By combining both sides (Equations~\eqref{eq:chap5_leftside_proof_thinning} and \eqref{eq:chap5_rightside_proof_thinning}) it follows that,
        for any $\varphi, \psi\in\mathcal{S}$:
        \[
            \int_{\RR}{\tilde \varphi(\omega)\tilde \psi(-\omega)\,\Gamma_p(\dd\omega)} = \int_{\RR}{\tilde {\varphi} (\omega) \tilde{\psi} (-\omega)\,(p^2\Gamma(\dd \omega) + p(1-p) M_1(\dd \omega))}\,.
        \]
        
        As this equality holds for any functions in the Schwartz space \textbf{Add reference}, then,
        \[\Gamma_p = p^2 \Gamma + p(1-p) M_1\,,\]
        and as $M_1 = m_1 \ell_{\RR}$,
        \[f_p(\omega) = p^2 f(\omega) + p(1-p) m_1\,,\]
        which achieves the proof.





    \section{Proof of Proposition~\ref{prop:chap5_identifiability_thinning}}\label{appendix:chap5_proof_identifiability_thinning}

    Let $\theta = (\mu, \alpha, \beta, p)$ and $\theta' = (\mu', \alpha', \beta', p')$ be two admissible parameter for model $\mathcal{Q}$.

    In order to prove that model $\mathcal{Q}$ is identifiable if and only if one of four parameters is fixed, 
    we will begin by retrieving the system of Equations~\eqref{eq:chap5_nonidentifiable_parameters}
    and verify that it defines an admissible parameter for $\mathcal{Q}$.
    Subsequently we will verify that by fixing one parameter, we retrieve the identifiability of the model.
    
    We begin by assuming that $f_\theta = f_{\theta'}$.
    By Equation~\eqref{eq:chap5_spectral_exponential}, this equality reads:
    \[\forall \omega \in \RR,\quad 
    \frac{\mu p }{1-\alpha}\left(1 + p \frac{\beta^2 \alpha (2-\alpha)}{\beta^2(1-\alpha)^2 + 4 \pi^2 \omega^2}\right) = 
    \frac{\mu' p' }{1-\alpha'}\left(1 + p' \frac{{\beta'}^2 \alpha' (2-\alpha')}{{\beta'}^2(1-\alpha')^2 + 4 \pi^2 \omega^2}\right)\,.
    \]

    Let us remark that a sufficient condition for both sides to be equal, 
    for all $\omega \in \RR$,
    is that the following system of equations is verified:

    \begin{equation}\label{eq:chap5_system_identifiability}
        \begin{cases}
            \frac{\mu p }{1-\alpha} = \frac{\mu' p'}{1-\alpha'}\\
            p \beta^2 \alpha(2-\alpha) = p' {\beta'}^2 \alpha'(2-\alpha')\\
            \beta (1-\alpha) = \beta' (1-\alpha')\,,
        \end{cases}
    \end{equation}
    as this would mean that the three constants (w.r.t. $\omega$) in both sides are equal.
    In fact this is also a necessary condition as the three equalities can be retrieved for $f(0)$, $\lim_{\omega\to+\infty}{f(\omega)}$ and $f(1)$
    and combining the equations.

    Let $\kappa\in(0, 1/p)$ and $p' = \kappa p$, which reduces the system of Equations~\eqref{eq:chap5_system_identifiability} to:

    \begin{subnumcases}{}
            \frac{\mu}{1-\alpha} = \frac{\mu' \kappa}{1-\alpha'}\label{eq:chap5_system_identifiability1}\\
            \beta^2 \alpha(2-\alpha) = \kappa {\beta'}^2 \alpha'(2-\alpha')\label{eq:chap5_system_identifiability2}\\
            \beta (1-\alpha) = \beta' (1-\alpha')\,,\label{eq:chap5_system_identifiability3}
    \end{subnumcases}

    As $\alpha > 1$, $\alpha'>1$, $\beta>0$ and $\beta'>0$, 
    Equation~\eqref{eq:chap5_system_identifiability3} can be expressed as:
    \[{\beta'}^2 = \frac{\beta^2(1-\alpha)^2}{(1-\alpha')^2}\,,\]
    and by replacing ${\beta'}^2$ in Equation~\eqref{eq:chap5_system_identifiability2}, we obtain:
    \begin{align}
        &\frac{\alpha (2-\alpha)}{(1-\alpha)^2} = \kappa \frac{\alpha' (2-\alpha')}{(1-\alpha')^2}\label{eq:chap5_alpha_equality} \\
        \iff\quad & \frac{1 - (1-\alpha)^2}{(1-\alpha)^2} = \kappa \frac{1 - (1-\alpha')^2}{(1-\alpha')^2}\nonumber \\
        \iff\quad & 1 + \frac{1}{\kappa}\left(\frac{1}{(1-\alpha)^2} - 1\right) = \frac{1}{(1-\alpha')^2}\nonumber\,,
    \end{align}
    and as $\alpha \in(0,1)$, the left-side term is positive and so:
    \[\alpha' = 1 - \frac{1}{\sqrt{1 + \frac{1}{\kappa}\left(\frac{1}{(1-\alpha)^2} - 1\right)}}\,.\]

    We can then obtain the explicit expressions of $\mu'$ and $\beta'$ with Equations~\eqref{eq:chap5_system_identifiability1} and \eqref{eq:chap5_system_identifiability3}
    providing the following system:
    
    \begin{equation}\label{eq:chap5_system_parameters}
        \begin{cases*}
            \mu' = \frac{\mu(1-\alpha')}{\kappa(1-\alpha)}\\
            \alpha' = 1 - \frac{1}{\sqrt{1 + \frac{1}{\kappa}\left(\frac{1}{(1-\alpha)^2} - 1\right)}}\\
            \beta' = \beta \frac{1-\alpha}{1-\alpha'}\\
            p' = \kappa p
        \end{cases*}
        \quad\iff\quad
        \begin{cases}
            \mu' = \frac{\mu}{\kappa(1-\alpha)\sqrt{1 + \frac{1}{\kappa}\left(\frac{1}{(1-\alpha)^2} - 1\right)}}\\
            \alpha' = 1 - \frac{1}{\sqrt{1 + \frac{1}{\kappa}\left(\frac{1}{(1-\alpha)^2} - 1\right)}}\\
            \beta' = \beta (1-\alpha) \sqrt{1 + \frac{1}{\kappa}\left(\frac{1}{(1-\alpha)^2} - 1\right)}\\
            p' = \kappa p\,.
        \end{cases}
    \end{equation}
    
    To verify that, for any $\kappa\in(0,1/p)$, parameter $\theta' = (\mu', \alpha', \beta', p')$ is an admissible parameter,
    we have to make sure that $\theta' \in \Theta = \RR_{>0}\times (0,1) \times \RR_{>0} \times (0,1)$.
    Let $\theta = (\mu, \alpha, \beta, p)\in\Theta$, $\kappa\in(0,1/p)$ and $\theta'$ defined by Equations~\eqref{eq:chap5_system_parameters}.
    For such a $\kappa$, $p'\in(0,1)$ is immediately verified.

    The rest of the conditions are verified if and only if: 
    \[\sqrt{1 + \frac{1}{\kappa}\left(\frac{1}{(1-\alpha)^2} - 1\right)} > 1 \iff \frac{1}{(1-\alpha)^2} - 1 > 0\\,.\]
    This is immediate as $\alpha\in(0,1)$ and so $\theta'\in\Theta$. 
    From Equations~\eqref{eq:chap5_system_parameters}
    we can see that $\theta'=\theta$ and we've proven that $f_\theta = f_{\theta'}$,
    so model $\mathcal{Q}$ is not identifiable.

    Lastly, let us show that if any of the four parameters is known then the model defined by the remaining triplet is identidiable.
    By considering the previously established equations (see Equations~\eqref{eq:chap5_alpha_equality} and \eqref{eq:chap5_system_parameters}), 
    we know that for $\theta = (\mu, \alpha, \beta, p)\in\Theta$ and $\theta' = (\mu', \alpha', \beta', p')\in\Theta$:
    \begin{equation}\label{eq:chap5_system_fixed}
        f_{\theta} = f_{\theta'} \quad\iff\quad 
        \begin{cases*}
            \mu' = \frac{\mu(1-\alpha')}{\kappa(1-\alpha)}\\
            \frac{\alpha (2-\alpha)}{(1-\alpha)^2} = \kappa \frac{\alpha' (2-\alpha')}{(1-\alpha')^2}\\     
            \beta' = \beta \frac{1-\alpha}{1-\alpha'}\\
            p' = \kappa p
        \end{cases*}\,.
    \end{equation}
    \begin{itemize}
        \item If $\mu = \mu'$, the system of Equations~\eqref{eq:chap5_system_fixed} reduces to:
        \begin{equation*}
            \begin{cases*}
                \kappa = \frac{(1-\alpha')}{(1-\alpha)}\\
                \kappa \frac{\alpha' (2-\alpha')}{(1-\alpha')^2} = \frac{\alpha (2-\alpha)}{(1-\alpha)^2}\\     
                \beta' = \frac{\beta}{\kappa}\\
                p' = \kappa p
            \end{cases*}\quad\iff\quad
            \begin{cases*}
                \kappa = \frac{(1-\alpha')}{(1-\alpha)}\\
                \frac{\alpha' (2-\alpha')}{(1-\alpha')} = \frac{\alpha (2-\alpha)}{(1-\alpha)}\\     
                \beta' = \frac{\beta}{\kappa}\\
                p' = \kappa p
            \end{cases*}\,.
        \end{equation*}

        As $\alpha\in(0,1)$ and $\alpha'\in(0,1)$, the second equation implies that $\alpha=\alpha'$ and so $\kappa=1$ and $\beta'=\beta$.

        \item If $\alpha = \alpha'$, the second equation in \eqref{eq:chap5_system_fixed} implies directly that $\kappa=1$ and so all other equalities follow.
        \item If $\beta = \beta'$, the third equation in \eqref{eq:chap5_system_fixed} implies that $\alpha=\alpha'$ and by the previous point all other equalities hold.
        \item If $p = p'$, $\kappa=1$ and the rest of equalities are verified.
        
        This shows that whenever one of the parameters is fixed, $f_\theta = f_{\theta'}$ implies $\theta = \theta'$. This achieves the proof.

    \end{itemize}



\end{subappendices}